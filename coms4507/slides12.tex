% !TeX spellcheck = en_US
% !TeX root = notes.tex
\section{Spectre and Meltdown Attacks}
\subsection{Spectre}
\begin{itemize}
	\item Attack on other programs or same programs memory using the Speculative Executor and CPU Cache
	\item Speculative Executor sees a branch and executes the code inside the branch regardless of the output of the branch condition. Result is stored on CPU Cache
	\item Once the code executes the branch condition, the data is removed from CPU Cache
	\item Scan the CPU Cache and check for values using reading time, if read time is low then the value is in the cache (and a the value of speculative execution)
	\item Speculative Execution can inject on a branch forcing it to speculative execute and read the result of that branch
	\item Can run in V8 engine (Javascript), KVM (Kernel Virtual Machine)
	\item Target program needs to run on same core (can flood the CPU with programs running 100\%)
\end{itemize}
\subsection{Meltdown}
\begin{itemize}
	\item More focused at using the Out-of-order execution and CPU Cache
	\item Out-of-order runs lines of code at the same time or before other lines which might take time
	\item Allows functions to be run that grant higher level permissions without having those permissions
	\item Requires no to minimal knowledge of the computer system
\end{itemize}
\subsection{Differences}
\begin{itemize}
	\item Spectre requires knowledge of the target program or host and how memory is handled and stored, Meltdown not as much
	\item Spectre is difficult to patch for and has significant speed impacts
	\item Spectre is used for accessing memory, Meltdown can access permissioned content
	\item Meltdown is more targeted to kernel space
\end{itemize}