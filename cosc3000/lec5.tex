% !TeX spellcheck = en_US
% !TeX root = notes.tex
\section{Time-series}
\paragraph{Nature of Time series data}
\begin{itemize}
	\item unidirectional
	\item discrete/continuous/(oridinal?)
	\item point-based/intervals
	\item can be nested
	\subitem measure something every day, another dataset of the same measurement is taken hourly
	\item can exhibit \textbf{cycles}
	\subitem days, week(end)s, months, seasons
	\item some ideas may apply to other data with spacing, frequency	
\end{itemize}
Time-series data can either discrete or continuous:
\begin{description}
	\item[Continuous:] temperature vs time
	\item[Discrete:] rainfall per day
\end{description}

\subsection{Time series periodicity}
\begin{description}
	\item[Fourier's theorem:] Any periodic function of time can be expressed as a sum of sine and cosine functions (i.e. as a Fourier series). Not periodic? Then you get a continuous Fourier integral rather than a discrete Fourier series.
	\item[Fourier transform:] Converts time-domain function to frequency-domain spectrum (Fourier series or integral, which we also call the Fourier transform).
	\item[Inverse Fourier transform:] Frequency-domain back to time-domain.
\end{description}
Method used on the computer is known as a \textbf{Fast Fourier Transform (FFT)}.