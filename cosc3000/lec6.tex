% !TeX spellcheck = en_US
% !TeX root = notes.tex
\section{Colour, light, and animation}
\subsection{Colour}
\begin{itemize}
	\item observation and interpretation of elements and relationships
	\item history and recommendations from cartography
	\item colour can:
	\subitem label
	\subitem measure
	\subitem represent reality
	\subitem emphasise
	\subitem enliven/decorate
	\item widespread
	\subitem but not trivial to get right	
\end{itemize}

\subsubsection{Rules}
\begin{itemize}
	\item good compromise: two hues, varying lightness
	\item keep strong colours for extremes
	\item not too many colours - 10 (paper), 15 (screen), 25 (greyscale)
	\item light/bright not next to white
	\item change hue with category,
	\item change saturation with rank/quantity
	\item avoid red/green contrasts	
\end{itemize}

\subsection{Animation}
\begin{itemize}
	\item attract attention, focus
	\item enjoyable, insightful
	\item enhance understanding
	\item great for complex objects
	\item worth the investment?
	\subitem time, effort, clarity (of graphics and info)	
\end{itemize}
What can be bad about animation?
\begin{itemize}
	\item It doesn't translate well to print
	\item It takes time and effort
	\item It can tie us to specific software
	\item It can make comparison harder - can you compare the current frame with a similar frame from 15 seconds ago?	
\end{itemize}
\subsubsection{Animation considerations}
\begin{itemize}
	\item Record/playback
	\begin{itemize}
		\item large/complex surfaces
		\item small set of stills, easily connected
	\end{itemize}
	\item Real time animation
	\begin{itemize}
		\item simple graphics objects
		\item user interaction
	\end{itemize}
	\item Other constraints
	\begin{itemize}
		\item computer speed/memory
		\item number of frames storable
		\item complexity of animation
		\item need for clarity not distraction (as always)
	\end{itemize}	
\end{itemize}

