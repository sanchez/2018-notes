% !TeX spellcheck = en_US
% !TeX root = notes.tex
\section{3D and 4+D Visualisation}
\begin{itemize}
	\item If we have 3 variables, we can plot them using 3 axes
	\item Gives us a 3D plot which is represented as a 2D image
	\item We may need cues to help identify the information
	\item More than 3D can be quite difficult to understand
	\subitem Quantitative methods become useful	
\end{itemize}


\section{Multivariate data analysis}
\textit{Yeah look, there was content but idk}

\section{Spatial statistics}
\begin{itemize}
	\item\textbf{Location:} What's happening at positions of interest?
	\item\textbf{Pattern:} Spatial arrangement of phenomena/events
	\item Analogous to previous descriptive stats, + space
\end{itemize}	

\subsection{Measures of dispersion}
Spectrum of dispersion (clustered $\rightarrow$ random $\rightarrow$ dispersed). Standard distance:
\[\sqrt{
	\frac{\sum_i{d_i^2}}
		{n}
}\]

\subsection{Nearest neighbour analysis}
\begin{itemize}
	\item Observed nearest-neighbour distance: $D_\text{obs} =$ mean distance to all points' nearest neighbours
	\item Expected nearest-neighbour distance: $D_\text{rand}=\frac{1}{2\sqrt{p}}$
	\subitem $p=$ density of points
	\item max clustering: $D_\text{obs}=0$
	\item max dispersion: $D_\text{obs}=\sqrt{\frac{2}{p\sqrt{3}}}$
	\item nearest-neighbour index $= \frac{D_\text{obs}}{D_\text{rand}} (\in [0, 2.15])$	
\end{itemize}
