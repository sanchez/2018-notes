\section{Visualising multidimensional data}
\subsection{Ternary plots}
\begin{itemize}
	\item Triangular ordination of \textit{proportions} of 3 components (i.e. $\sum=1$)
	\item tricky but worth the effort
	\item axes read parallel to each zero (center = 33\% each)	
\end{itemize}

\subsection{Scatterplot matrix}
\begin{itemize}
	\item tiled 2d visualisation of multiple variables
\end{itemize}

\subsection{Coplots}
\begin{itemize}
	\item conditional on one variable
	\subitem slice through data
	\subitem slices trade-off range and data
	\subitem fit curves to see dependency
	\subitem can also graph residuals etc.	
\end{itemize}

\subsection{3d point plots}
\begin{itemize}
	\item scatterplot in 3d
	\item used to seeing a 3d object in 2d
	\item if not too many points can use a stem plot	
\end{itemize}

\subsubsection{Interpolating}
\begin{itemize}
	\item if irregular x/y data
	\item create lattice covering range of x and y
	\item interpolate between z-values	
\end{itemize}

\subsubsection{viewing/presentation choices}
\begin{itemize}
	\item grid density
	\begin{itemize}
		\item sparse = lose surface definition
		\item dense = lose depth perception
	\end{itemize}
	\item axis ratios
	\begin{itemize}
		\item as data if axes share same units
		\item otherwise 1-1-1
		\item or banked z-axis?
	\end{itemize}
	\item projection
	\begin{itemize}
		\item perspective: good for real objects (but distortion)
		\item orthogonal: fixed lengths $\rightarrow$ data analysis (but front/back confusion)
	\end{itemize}
	\item orientation (azimuth and elevation)
	\begin{itemize}
		\item multiple rotations helpful (around z-axis, in equal steps)
	\end{itemize}
	\item other effects
	\begin{itemize}
		\item box
		\item ticks
		\item curtain
		\item show/hide rear wires
		\item colour
	\end{itemize}	
\end{itemize}
All involve trade-offs
\begin{itemize}
	\item help or hindrance
	\item should highlight data, not effect itself	
\end{itemize}

\subsubsection{colour representation}
use of colour
\begin{itemize}
	\item none
	\item height-related
	\item gradient-related	
\end{itemize}
