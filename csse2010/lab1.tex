% !TeX spellcheck = en_US
% !TeX root = notes.tex
\subsection{Conversions}
Easiest to convert from most formats to binary then to the desired format.

\subsubsection{Octal}
\begin{description}
	\item[From Binary:] Group bits into series of 3 and then convert to decimal (0b010 = 02)
	\item[To Binary:] Convert each octal number to binary and append
\end{description}

\subsubsection{Hex}
\begin{description}
	\item[From Binary:] Group bits into series of 4 and then convert to hex with overflow being apart of the alphabet (0b1100 = 0xC)
	\item[To Binary:] Convert each hex number to binary and append
\end{description}

\subsubsection{Decimal}
\begin{description}
	\item[From Binary:] Add together the powers of two at each position $n$ ($0b1010 = 2^3 + 2^1 = 10$)
	\item[To Binary:] Starting with LSB, divide by 2 with the remainder being bit value at position. ($9 = 9/2 = 4 rem 1, 4/2 = 2 rem 0, 2/2 = 1 rem 0, 1/2 = 0 rem 1. \therefore 9 = 0b1001$)
\end{description}


\subsection{Negative Numbers}
\subsubsection{Signed Magnitude}
Leftmost bit is the sign bit, true is negative and false is positive

\subsubsection{One's Complement}
Leftmost bit = sign-bit (as per signed magnitude), true is negative and false is positive. If negative all bits are inverted

\subsubsection{Two's Complement}
MSB signifies if negative, true is negative and false is positive. To negate invert all bits and add decimal 1.
\begin{leftbar}
	Allows addition without requiring conversion	
\end{leftbar}


\subsubsection{Excess $2^{m-1}$}
e.g. for 8 bits, excess-128. Add 128 to the original bit and convert to binary