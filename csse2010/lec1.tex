% !TeX spellcheck = en_US
% !TeX root = notes.tex
\section{Bits, Bytes and Binary}
\subsection{Structured Computer Organization}
\begin{description}
	\item[Level 5:] Problem-oriented language level
	\item[Level 4:] Assembly language level
	\item[Level 3:] Operating system machine level
	\item[Level 2:] Instruction set architecture level
	\item[Level 1:] Microarchitecture level
	\item[Level 0:] Digital Logic level
\end{description}

\subsection{Unsigned Number in Binary}
Each bit position has a value $\rightarrow 2^n$ (starting at zero). Add all values of the positions together and that's unsigned value.

\subsection{Least and Most Significant Bits}
\begin{description}
	\item[Most Significant Bit (MSB):] Bit that's worth the most, the left-most bit
	\item[Least Significant Bit (LSB):] Bit that's worth the least, the right-most bit
\end{description}

\subsection{Radices}
\begin{itemize}
	\item \textbf{Radix:} number system base
	\item A radix-k number system
	\subitem $k$ different symbols to represent digits 0 to $k-1$
	\subitem Value of each digit is (from the right) $k^0, k^1, k^2, k^3, \ldots$
	\item Often convenient to deal with
	\subitem\textbf{Octal} (radix-8) - Symbols: 0, 1, 2, 3, 4, 5, 6, 7
	\subsubitem\textit{One octal digit corresponds to 3 bits}
	\subitem\textbf{Hexadecimal} (radix-16) - Symbols: 0, 1, 2, 3, 4, 5, 6, 7, 7, 8, 9, A, B, C, D, E, F
	\subsubitem\textit{One hexadecimal digit corresponds to 4 bits (useful)}
\end{itemize}

\subsubsection{Radix Identification}
\begin{itemize}
	\item Hexadecimal
	\subitem Leading 0x (C, Atmel AVR)
	\subitem Trailing h (Some assembly languages)
	\subitem Leading \$ (Atmel AVR Assembly)
	
	\item Octal
	\subitem Leading 0 (C, Atmel AVR)
	\subitem Trailing q (Some assembly languages)
	\subitem Leading @ (Some assembly languages)
	
	\item Binary
	\subitem Leading 0b (Atmel AVR Assembly, Some C)
	\subitem Trailing b (Some assembly languages)
	\subitem Leading \% (some assembly languages)
\end{itemize}