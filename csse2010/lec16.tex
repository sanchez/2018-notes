\section{Interrupts}
CPU receiving interrupt causes it to execute software to handle the interrupt
\begin{itemize}
	\item Software routine is called \textbf{interrupt handler} or \textbf{interrupt service routine (ISR)}
\end{itemize}

\subsection{Finding the Interrupt Handler}
Look in the \textbf{interrupt vector table} which contains either a table of addresses or table of jump instructions

\subsubsection{Transparency}
When ISR finishes, computer is in the same state as it was before the interrupt. Save the status register and any other registers that will be used by the interrupt.

\subsubsection{Interrupt Priority}
Priority only occurs when two priorities are triggered at relatively the same time. Highest priority is the one with the lower vector value

\subsection{Traps}
Traps are \textbf{software interrupts} caused by events in software.
\begin{itemize}
	\item Overflow, Divide by zero, Undefined opcode
	\item Traps save continually checking for errors
	\item Trap handlers (service routines) don't always return to the original program
\end{itemize}
