\section{Assembly}
Statements have 4 parts:
\begin{description}
	\item[Labels] needed so statements can be jumped to, and so that data can be referenced
	\item[Opcode] has symbolic abbreviation for an instruction opcode
	\item[Operand] field specifies addresses and registers used as operands of the instruction
	\item[Comments] field is space for programmer to put helpful explanations
\end{description}
A command (directive) to the assembler itself! Preceded with a ``.'', called \textbf{Pseudo-instruction} or \textbf{directive}

\subsection{Memory Segments}
\begin{itemize}
	\item Different types of memory are known as \textbf{segments} to the assembler
	\item Assembler directives enable code/data to be placed into different segments
	\item AVR has
	\begin{itemize}
		\item Data segment (RAM)
		\subitem Can't place values here, just reserve space (for variables)
		\item Code segment (Flash)
		\subitem Can place program code or constant data here
		\item EEPROM Segment	
		\subitem Can place constants here
	\end{itemize}
\end{itemize}

\subsection{Pseudo-instructions}
\begin{itemize}
	\item \texttt{.byte}: Reserve space; only allowed in \textbf{dseg} (RAM)
	\item Segment directives \texttt{.cseg} and \texttt{.dseg} allow the text and data segments to be built up in pieces
	\item \texttt{.db}: Initialise constant in code or EEPROM segment
	\item \texttt{.dw}: As above but defines a 16-bit word
	\item \texttt{.def}: Make a definition (for registers only)
	\item \texttt{.device}: Specify the exact processor that this program is designed for (\texttt{.device ATmega324A})
	\item \texttt{.include}: Include a file
	\item \texttt{.exit}: Stop processing this file
	\item \texttt{.equ}: Equate (not changeable)
	\item \texttt{.set}: Equate (but changeable)
	\item \texttt{.org}: Set location counter - i.e. address (in any segment)
\end{itemize}
