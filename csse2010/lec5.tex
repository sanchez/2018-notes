% !TeX spellcheck = en_US
% !TeX root = notes.tex
\section{Flip-flops}
\subsection{D Flip Flop}
\begin{itemize}
	\item\textbf{D} is input
	\item\textbf{Q} is output
	\item\textbf{CLK} (clock) is control input	
\end{itemize}
Q copies the value of D (and remembers it) whenever CLK goes from 0 to 1 (\textbf{rising edge}).

\subsubsection{Characteristic Table}
\textbf{Characteristic table} defines operation of flip-flop in tabular form
\begin{table}[H]
	\centering\caption{D Flip-Flop Characteristic Table}
	\begin{tabular}{c|c}
		D & Q(t+1)\\\hline
		0 & 0\\
		1 & 1
	\end{tabular}
\end{table}

\subsection{Flip-Flops Vs Latches}
\begin{itemize}
	\item The last few slides show \textbf{latches}
	\subitem These are \textbf{level-triggered} devices
	\item Remember we want to capture the input value at rising \textbf{edge} (a short instant)!
	\item Any devices based on edges are referred to as \textbf{flip-flops}
	\subitem These are \textbf{edge-triggered} devices	
\end{itemize}
