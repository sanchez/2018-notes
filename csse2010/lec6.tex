% !TeX spellcheck = en_US
% !TeX root = notes.tex
\section{Shift Registers}
\subsection{Combinational vs Sequential Circuits}
\begin{itemize}
	\item\textbf{Combinational} Circuits
	\begin{itemize}
		\item Logic gates only (no flip-flops)
		\item Output is uniquely determined by the inputs
		\subitem i.e. you'll always get the same output for a given set of inputs
	\end{itemize}
	\item\textbf{Sequential} Circuits
	\begin{itemize}
		\item Include flip-flops
		\item Output determined by current inputs and current \textbf{state} (values in the flip-flops)
		\item Output can change when clock ``ticks'' (rising edge)	
	\end{itemize}
\end{itemize}
\subsubsection{Sequential Circuits}
\begin{itemize}
	\item\textbf{State} is value stored in flip-flops
	\item Output depends on input and state
	\subitem or sometimes just the state
	\item Next state depends on inputs and state	
\end{itemize}

\subsubsection{Synchronous Sequential Circuit}
\begin{itemize}
	\item Storage elements (flip-flops) can only change at discrete instants of time	
\end{itemize}

\subsection{Registers}
\begin{itemize}
	\item A \textbf{register} is a group of flip-flops
	\subitem $n$-bit register consists of $n$ flip-flops capable of storing $n$ bits
	\item A register is a sequential circuit \textit{without} any combinational logic
\end{itemize}

\subsubsection{Shift Register}
A shift register is a register which is capable of shifting its binary information in one or both directions