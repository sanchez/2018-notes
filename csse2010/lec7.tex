% !TeX spellcheck = en_US
% !TeX root = notes.tex
\section{Counters}
\begin{itemize}
	\item A \textbf{counter} is a multi-bit register that goes through a determined sequence of states (values) upon the application of input pulses
	\item A counter which follows binary number sequence is a \textbf{binary counter}
	\subitem $n$-bit binary counter has $n$ flip-flops and can count from 0 to $2^n-1$
\end{itemize}

\begin{note}{State}
	\begin{itemize}
		\item Values stored in the flip-flops can be considered the \textbf{current state} of the circuit
		\item D inputs to the flip-flops are the \textbf{next state}
		\item D inputs are some function of the current state and inputs	
	\end{itemize}
\end{note}


\paragraph{Key Points}
\begin{itemize}
	\item Next state is a function of previous state (and possibly inputs)
	\item Count sequence can be binary numbers but does not have to be
	\subitem If it is, counter is a \textbf{binary counter}
	\item Circuits are \textbf{synchronous}
	\subitem All flip-flops have the same clock	
\end{itemize}

