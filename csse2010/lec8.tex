% !TeX spellcheck = en_US
% !TeX root = notes.tex
\section{State Machines}
\begin{itemize}
	\item Sequential circuits can also be called
	\subitem state machines
	\subitem finite state machines (FSMs)
	\item State machine has
	\begin{itemize}
		\item Finite number of possible states
		\item Only one \textbf{current state}
		\item Can \textbf{transition} to other states based on inputs and current state
	\end{itemize}
\end{itemize}

\begin{note}{Types of State Machines}
	\begin{description}
		\item[Mealy Machines:] Outputs depend on current state and inputs
		\item[Moore Machines:] Outputs depend only on current state (flip-flop values)
		\subitem Outputs can only change when state changes
	\end{description}	
\end{note}

\subsection{State diagram}
\begin{figure}[H]
	\begin{tikzpicture}[node distance=2cm,auto,every place/.style={draw}]
		\node [place] (S1) {\begin{tabular}{c}S1\\00\end{tabular}};
		\coordinate[node distance=1.5cm,left of=S1] (left-S1);
		\draw[->,thick] (left-S1) -- (S1);
		
		\node [place] (S2) [right=of S1] {\begin{tabular}{c}S2\\01\end{tabular}};
		\path[->] (S1) edge [bend left] node {$a$} (S2);
		\path[->] (S2) edge [bend left] node {$b$} (S1);
		\path[->] (S2) edge [loop above] node {$\bar{a}$} ();
		\path[->] (S1) edge [loop above] node {$\bar{a}$} ();
	\end{tikzpicture}	
	\centering
	\caption{Example single input state diagram}
\end{figure}
\textit{Note: I couldn't figure out how to add the line that is meant to go between the state label and the state number}

\subsubsection{Completeness}
Each possible combination of inputs should be addressed \textbf{exactly once} for each state. i.e. transition arrows from each state must encompass all possibilities (exactly once)

\subsection{State tables}
\begin{itemize}
	\item State diagrams can also be represented in a state table	
\end{itemize}

\begin{table}[H]
	\centering
	\caption{Example State Table}
	\begin{tabular}{cc|c|cc}
		\textbf{Current State} & \textbf{Input $U$} & \textbf{Next State} & \multicolumn{2}{c}{\textbf{Outputs}}\\
		&&& $Q_1$ & $Q_2$\\\hline
		S0 & 0 & S3 & 0 & 0\\
		S0 & 1 & S1 & 0 & 0\\
		S1 & 0 & S0 & 0 & 1\\
		S1 & 1 & S2 & 0 & 1\\
		S2 & 0 & S1 & 1 & 0\\
		S2 & 1 & S3 & 1 & 0\\
		S3 & 0 & S2 & 1 & 1\\
		S3 & 1 & S1 & 1 & 1
	\end{tabular}
\end{table}

\begin{table}[H]
	\centering
	\caption{Two-dimensional state table}
	\begin{tabular}{c|cc|cc}
		\textbf{Current} & \multicolumn{2}{c}{Next State} & \multicolumn{2}{c}{Outputs}\\
		\textbf{State} & $\bar{U}$ & $U$ & $Q_1$ & $Q_0$\\\hline
		S0 & S3 & S1 & 0 & 0\\
		S1 & S0 & S2 & 0 & 1\\
		S2 & S1 & S3 & 1 & 0\\
		S3 & S2 & S0 & 1 & 1
	\end{tabular}
\end{table}

\subsection{State encoding}
\begin{itemize}
	\item Must encode each state into flip-flop values
	\item Choose
	\subitem Number of flip-flops
	\subitem Bit patterns that represent each state
	\item Ideally, choose state encoding to make combinational logic simple, for both
	\subitem Output logic
	\subitem Next state logic	
\end{itemize}