\section{ALUs and Memory}
\subsection{Parts of a CPU}
\begin{itemize}
	\item\textbf{Control Unit}
	\subitem Fetches instructions from memory, makes the ALU and the registers perform the instruction
	\item\textbf{ALU}
	\subitem Performs arithmetic and logical operations
	\item\textbf{Registers}
	\subitem High speed memory -- stores temporary results and control
\end{itemize}

\subsection{Registers}
Different types of registers
\begin{itemize}
	\item\textbf{Program Counter Register}
	\subitem Stores the memory address of the next instruction to be fetched
	\item\textbf{Instruction Register}
	\subitem Contains the current instruction
	\item\textbf{General Purpose Registers}
	\subitem Contains data to be operated on (e.g. data read from memory), results of operations, ...
	\subitem Width is CPU word size
	\subitem Sometimes called the \textbf{register file}
\end{itemize}

\subsection{Buses}
\textbf{Bus} = Common pathway (collection of ``wires'') connecting parts of a computer\\
Characteristics:
\begin{itemize}
	\item Can be \textbf{internal} to CPU (e.g. ALU to registers)
	\item Can be \textbf{external} to CPU (e.g. CPU to memory)
	\item Buses have a \textbf{width} -- number of bits that can be transferred together over a bus
	\subitem May not always be the same as the word size of the computer
\end{itemize}

\begin{note}{Arithmetic Logic Unit (ALU)}
Does more than adding... \textbf{Function / control} input dictates the operation that the ALU is to perform, e.g.
\begin{itemize}
	\item Addition
	\item Increment (+1)
	\item Subtraction
	\item Bitwise AND
	\item Bitwise OR	
\end{itemize}
Like adders, ALUs can be made up from 1-bit slices
\end{note}

\subsection{Data Path}
\begin{itemize}
	\item Operands come from register file
	\item Result written to register file
	\item Implements routine instructions such as arithmetic, logical, shift
	\item Width of registers/buses is the CPU word size	
\end{itemize}
