% !TeX spellcheck = en_US
% !TeX root = notes.tex
\section{Communication}
\begin{itemize}
	\item Interprocess communication in DS is based on message exchange
	\item Low level (transport) message passing does not provide distribution transparency
	\subitem higher level models are needed
	\item Distributed application need a variety of communication semantics
	\begin{itemize}
		\item Remote Procedure Call (RPC)
		\item Message-Oriented Middleware (MOM)
		\item Data streaming
	\end{itemize}	
\end{itemize}

\subsection{Layered Protocols}
Layers, interfaces, and protocols in the \textbf{Open Systems Interconnection Reference Model} (OSI)
\begin{description}
	\item[Application:] Application protocol
	\item[Presentation:] Presentation protocol
	\item[Session:] Session protocol
	\item[Transport:] Transport protocol
	\item[Network:] Network protocol
	\item[Data link:] Data link protocol
	\item[Physical:] Physical protocol
\end{description}

\subsubsection{Middleware Layer}
Provides common protocols that can be used by many different applications
\begin{itemize}
	\item Security protocols
	\item Transaction protocols
	\item High-level communication models (e.g. RPC, message queuing services, etc)	
\end{itemize}

\subsection{Types of Communication}
\subsubsection{Synchronicity in Communication}
\begin{leftbar}
	\textbf{Persistent communication} - message is stored in the system until receiver becomes active	
\end{leftbar}

\paragraph{Persistent asynchronous communication.} A will send a message to B and not wait for response back
\paragraph{Persistent synchronous communication.} A sends a message and wait's for B to respond before continuing execution
\paragraph{Transient asynchronous communication.} A sends a message, B will only receive the message if it is running
\paragraph{Receipt-based transient synchronous communication.} A sends a message, B receives the message if it is active and returns an acknowledgement
\paragraph{Delivery-based transient synchronous communication at message delivery.} A sends a message and waits for the acknowledge from B, B will receive the message and send an acknowledge when it starts on the job
\paragraph{Response-based transient synchronous communication.} A sends a message and waits for the acknowledge from B, B will receive the message and send an acknowledge when it has finished the job

\begin{note}{Parameter Passing}
	\textbf{Parameter marshalling:} There's more than just wrapping parameters into a message. Client and Server:
	\begin{itemize}
		\item have different data representations
		\item have to agree on the same encoding (basic/complex data values)
		\item Interpret data and transform them into machine-dependent representation	
	\end{itemize}
\end{note}

\subsection{Parameter Specification and Stub Generation}
\begin{itemize}
	\item Parameters passed by value do not pose problems
	\item Parameters passed by reference - some partial solutions exist (e.g. a small array can be sent to the server)
	\item Interfaces (procedures) are often specified in Interface Definition Language (IDL) and compiled into stubs	
\end{itemize}

\subsection{Message-Oriented communication}
\begin{itemize}
	\item RPC (and RMI) are \textbf{synchronous} and \textbf{transient} (both sender and receiver have to be active
	\item Transport level communication like TCP and UDP also \textbf{require that the sender and receiver are active} (transient communication)
	\item Some applications require that messages are kept in the system until the receiver becomes active (e-mail is one example, but there are many more applications of this type)	
\end{itemize}
\subsubsection{Asynchronous Communication}
\begin{itemize}
	\item MOM -- Message-Oriented Middleware
	\item Can store messages in the system
	\item Applications communicate by inserting messages in queues
	\item Messages are delivered by an overlay network of application layer servers (routers)
	\item Each application has its own private queue to which other applications can send messages
	\item There is no guarantee on when the message will be delivered and that it will be read	
\end{itemize}

\subsubsection{Message-Queueing Model}
\begin{description}
	\item[Put:] Append a message to a specific queue -- non-blocking
	\item[Get:] Block until the specified queue is nonempty, and remove the first message
	\item[Poll:] Check a specified queue for messages, and remove the first. Never block
	\item[Notify:] Install a handler to be called when a message is put into the specified queue
\end{description}

\subsubsection{Message brokers}
\begin{itemize}
	\item Main role of brokers is to transform messages from sender's format to receiver's format
	\item This functionality can be generalised to brokers matching applications based on messages
	\subitem Applications \textit{publish} messages
	\subitem Applications \textit{subscribe} for messages	
\end{itemize}

\begin{note}{MCA: Message Channel Agent}
	Responsible for checking a message, wrapping it, and sending it to associated receiving ends	
\end{note}

\subsubsection{Channels}
\begin{description}
	\item[Transport type:] Determines the transport protocol to be used
	\item[FIFO delivery:] Indicates that messages are to be delivered in the order they are sent
	\item[Message length:] Maximum length of a single message
	\item[Setup retry count:] Specifies maximum number of retries to start up the remote MCA
	\item[Delivery retries:] Maximum times MCA will try to put received message into queue
\end{description}

\subsubsection{Message Transfer}
\begin{description}
	\item[MQopen:] Open a (possibly remote) queue
	\item[MQclose:] Close a queue
	\item[MQput:] Put a message into an opened queue
	\item[MQget:] Get a message from a (local) queue	
\end{description}

\subsection{Transient Communication}
\subsubsection{Berkeley Sockets}
\begin{description}
	\item[Socket:] Create a new communication end point
	\item[Bind:] Attach a local address to a socket
	\item[Listen:] Announce willingness to accept connections
	\item[Accept:] Block caller until a connection request arrives
	\item[Connect:] Actively attempt to establish a connection
	\item[Send:] Send some data over the connection
	\item[Receive:] Receive some data over the connection
	\item[Close:] Release the connection
\end{description}

\subsection{Actor model for communication}
\begin{itemize}
	\item The actor model adopts the philosophy that \textit{everything is an actor}
	\item Can respond to a message it receives, can send a finite number of messages to other actors
	\item Enabling asynchronous communication and control structures as patterns of passing messages
	\item Resemble the Enterprise application integration framework	
\end{itemize}

\subsection{Stream-oriented communication}
\begin{itemize}
	\item Some applications need to exchange time-dependent information e.g. video, audio (continuous media)
	\item Previously discussed models do not consider time
	\item In continuous media the temporal relationship between different data items is essential	
\end{itemize}

\subsection{Different transmission modes}
\begin{itemize}
	\item Asynchronous transmission mode
	\subitem Sequential transmission without restrictions on when data is to be delivered (e.g. file transfer)
	\item Synchronous transmission mode
	\subitem Max end-to-end delay for each unit in a data stream (e.g. sensor sample temperature)
	\item Isochronous transmission mode (streams)
	\subitem Data transfer bounded by maximum and minimum end-to-end delay (bounded jitter e.g. audio/video)	
\end{itemize}

\subsection{Streams}
Streams can be simple or complex (i.e. can include several related substreams)
\begin{itemize}
	\item Unidirectional (source $\rightarrow$ sinks)
	\item Simple (single flow of data e.g. audio, video)
	\item Complex (stereo audio, movie)	
\end{itemize}

\subsubsection{Streams and QoS}
Streams need timely delivery of data
\begin{itemize}
	\item Non functional requirements (time, bandwidth, volume, reliability) are expressed as Quality of Service (QoS)
	\item There are many models for QoS specifications: e.g. IntServ, DiffServ, MPLS
	\item In IntServ (Integrated Services), QoS is specified as a flow specification -- flow reservation
	\item In DiffServ (Differentiated Services), QoS is specified for a class (e.g. expedited forwarding class)	
\end{itemize}
Properties for Quality of Service:
\begin{itemize}
	\item The required bit rate at which data should be transported (application specific)
	\item The maximum delay until a session has been set up (when to start sending data)
	\item The maximum end-to-end delay
	\item The maximum delay variance, or jitter
	\item The maximum round-trip delay	
\end{itemize}

\subsubsection{Stream Synchronisation}
\begin{itemize}
	\item Given	a complex stream, how are substreams synchronised?
	\item Synchronisation takes place at the level of data units (e.g. synchronise two streams only between data units)
	\item Issues which need to be considered:
	\subitem Mechanisms for synchronising streams
	\subitem Distribution of those mechanisms
\end{itemize}

\subsubsection{Distribution of synchronisation mechanisms}
\begin{itemize}
	\item The receiving side needs to have a complete synchronisation specification
	\item Synchronisation specification can be multiplexed together with other substreams into a complex stream (and is demultiplexed after receiving)	
\end{itemize}

\subsubsection{Multicast communication}
\begin{itemize}
	\item Application level multicasting -- setting a path for information dissemination
	\begin{itemize}
		\item Nodes (applications) organise into an overlay network
		\item Overlay network disseminates information
		\item Network layer routing is independent of the overlay (communication may not be optimal)
	\end{itemize}
	\item Overlay organisation
	\begin{itemize}
		\item Nodes may organise themselves into a tree, or
		\item Nodes organise themselves into a mesh network
	\end{itemize}
\end{itemize}

\subsubsection{Gossip-based data dissemination}
\begin{itemize}
	\item Epidemic protocols are often used for data dissemination
	\begin{itemize}
		\item There is no central component which coordinates dissemination
		\item Information is propagated using local information only
	\end{itemize}
	\item Node is \textit{infected} if is has data which it wants to spread
	\item Node which has not seen this data is \textit{susceptible}	
\end{itemize}

\begin{note}{Information Dissemination Models}
	\begin{itemize}
		\item Anti-entropy propagation model
		\begin{itemize}
			\item Node $P$ picks another node $Q$ at random
			\item Subsequently exchanges updates with $Q$
		\end{itemize}	
		\item Approaches to exchanging updates
		\begin{itemize}
			\item $P$ only pushes its own updates to $Q$
			\item $P$ only pulls in new updates from $Q$
			\item $P$ and $Q$ send updates to each other
		\end{itemize}
		\item One variant is the ``gossiping'' protocol
	\end{itemize}
\end{note}
