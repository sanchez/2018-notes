% !TeX spellcheck = en_US
% !TeX root = notes.tex
\section{Naming}
\subsection{Names, Identifiers and Addresses}
\begin{itemize}
	\item Names refer to entities
	\subitem e.g. processes, users, mailboxes, network connections
	\item Special types of names exist:
	\begin{itemize}
		\item Addresses
		\item Identifiers
		\item Human-friendly names
	\end{itemize}
\end{itemize}

\subsubsection{Addresses}
\begin{itemize}
	\item To use (operate on) an entity you need an access point
	\item The name of an access point is an \textit{address}
	\item Entities can have more than one access point at a time
	\item An entity may change access points e.g. when it changes location
	\item \textit{Location-independent names} are not tied to an address
\end{itemize}

\subsubsection{Identifiers}
Special type of name that:
\begin{itemize}
	\item Refers to at most one entity
	\item Always refers to the same entity
	\item Are limited to one per entity
\end{itemize}

\subsubsection{Human-friendly names}
Generally represented as a character string e.g. \texttt{www.news.com.au} or \texttt{/root/ryan/slides}

\subsection{Flat Naming}
Random bit strings (name doesn't help locate access point)
\begin{itemize}
	\item Names must be resolved to address
	\item There are many name resolution approaches for flat names:
	\begin{itemize}
		\item Broadcasting (or multicasting) approaches
		\item Forwarding pointers
		\item Home-based approaches (e.g. Mobile IP)
		\item Distributed Hash Tables (DHTs)
		\item Hierarchical approaches
	\end{itemize}
\end{itemize}

\subsubsection{Broadcasting}
Broadcasting (or multicasting) the ID requesting the entity to return its current address. e.g. ARP (Address Resolution Protocol)

\subsubsection{Forwarding Pointers}
\begin{itemize}
	\item When entity moves, it leaves a pointer to its new location
	\item To find entity, must follow trail of pointers to entity's current location
	\item Problems:
	\begin{itemize}
		\item Can get long chains of pointers (not scalable)
		\item If chain breaks, can't find entity
	\end{itemize}
	\item Potential solution:
	\subitem Stub Scion Pair (SSP) chains	
\end{itemize}

\subsubsection{Home-Based Approach}
\begin{itemize}
	\item Approach designed for mobile entities
	\item Entities have home location that keeps track of entity's current location
	\item Example of home-based approach is Mobile IP:
	\begin{itemize}
		\item Mobile entity has fixed IP address
		\item All traffic to mobile entity goes to entity's Home Agent (home location)
		\item Home Agent forwards traffic to Care-Of-Address (mobile entity's IP in its current network)
		\item Whenever mobile entity changes network, it gets new Care-Of-Address, which it registers with Home Agent
	\end{itemize}	
\end{itemize}

\subsubsection{Distributed Hash Tables}
\begin{itemize}
	\item Consist of many distributed nodes
	\item Map data to a key value using a hash function
	\item Each node is responsible for key values (and the associated data) in a particular range
	\item Have fast lookup times	
\end{itemize}

\subsubsection{Hierarchical Approach}
\begin{itemize}
	\item Network is divided into \textbf{domains}, with each domain subdivided into smaller subdomains
	\item \textbf{Leaf domain} is lowest level domain (typically LAN or Cell in mobile phone network)
	\item Each domain has a \textbf{directory node} that keeps track of entities in that domain
	\item Directory nodes for leaf domains store address of entities in that domain
	\item Directory nodes for non-leaf domains store reference to lower-level domain containing entities	
\end{itemize}

\subsection{Structured Naming}
\begin{itemize}
	\item Flat names are not convenient for humans
	\item Structured names:
	\begin{itemize}
		\item Composed from simple human-readable names
		\item Generally supported by naming systems
		\item Names are organised into a \textbf{name space}
	\end{itemize}	
\end{itemize}

\subsubsection{Name Spaces}
\begin{itemize}
	\item Name spaces for structured names are represented as a labelled, directed naming graph in which:
	\begin{itemize}
		\item A \textbf{leaf node} represents a (named) entity
		\item A \textbf{directory node} is an entity that refers to other nodes
		\subitem Outgoing edge is represented as (edge label, node identifier)
		\item \textbf{Root node} has only outgoing edges
	\end{itemize}
	\item Naming graphs are usually directed acyclic graphs
	\item Each path in a naming graph can be referred to by a sequence of edge labels separated by a special character	
\end{itemize}

\subsubsection{Name Resolution in a Name Space}
\begin{itemize}
	\item Need to know how and where to start $\rightarrow$ need closure mechanism
	\item Closure mechanism deals with selecting initial node in a name space from which name resolution is to start
	\item Closure mechanisms are often implicit	
\end{itemize}

\subsubsection{Linking and Mounting}
\begin{itemize}
	\item Aliases
	\begin{itemize}
		\item Another name for the same entity
		\item Symbolic links in UNIX enable more than one path for a file
	\end{itemize}
	\item Mounting
	\begin{itemize}
		\item Used to merge different name spaces transparently
	\end{itemize}	
\end{itemize}

\subsubsection{Name Resolution}
\begin{itemize}
	\item Structured names resolved using naming service (implemented by name servers)
	\item Naming service allows addition, removal and lookup of names
	\item DS naming services are distributed
	\item Large-scale distributed naming services are usually hierarchical:
	\subitem\textbf{Global level:} high-level directory nodes
	\subitem\textbf{Administrational level:} mid-level directory nodes grouped into separate administrations
	\subitem\textbf{Managerial level:} low-level directory nodes within a single administration
	\item The Domain Name System is a good example of a structured name resolution mechanism
\end{itemize}

\subsubsection{DNS}
\begin{itemize}
	\item Large distributed name service used by Internet
	\item DNS name space is hierarchical
	\item DNS labels use alphanumeric character strings separated by a ``.''
	\item A path name in DNS is called a domain name
	\item DNS is primarily used to lookup IP addresses for hosts and mail servers
	\item Client (e.g. browser) contacts name resolver
	\item Name resolver uses either \textbf{iterative} or \textbf{recursive} approach
	\item Iterative name resolution:
	\begin{enumerate}
		\item Name resolver contacts name servers for help
		\item Starts at root name server
		\item Each name server resolves as much of name as it can before referring name resolver to another name server who knows more
		\item Process repeats until name is fully resolved, or cannot be resolved anymore
	\end{enumerate}	
	\item Recursive name resolution:
	\begin{enumerate}
		\item Name resolver sends name to root name server
		\item Intermediate result not passed to client, rather it is sent to next name server
		\item Process repeats until name is fully resolved, then servers pass fully resolved name back
		\item Root name server passes fully resolved name to name resolver
	\end{enumerate}
\end{itemize}
Comparison between Iterative and Recursive:
\begin{itemize}
	\item Recursive approach has higher performance demand on name servers
	\item Recursive approach can use result caching more effectively
	\item Iterative approach has higher communication costs	
\end{itemize}

\subsection{Attribute-based Naming}
\begin{itemize}
	\item Each entity is described by \textit{(attribute, value)} pairs
	\item Attribute-based queries:
	\subitem Users specify attributes they are looking for
	\subitem Naming system should return one (or more) entities with the specified attributes
	\item Attribute-based naming systems are commonly known as \textbf{directory services}
	\item Lightweight Directory Access Protocol (LDAP) is a common directory service	
\end{itemize}

\subsubsection{LDAP}
\begin{itemize}
	\item Derived from OSI X.500 directory service
	\item LDAP directory service stores \textit{directory entries}
	\item Directory entries consist of (attribute, value) pairs
	\item Directory Information Base (DIB) is collection of all directory entries in an LDAP service
	\item Each LDAP directory entry has globally unique name (Directory Information Tree) based on hierarchy of naming attributes
	\subitem e.g. /C=NL/O=Vrije/OU=Comp.Sc	
\end{itemize}
\begin{table}[H]
	\centering
	\caption{Simple Example of LDAP directory entry}
	\begin{tabular}{lc|l}
		\textbf{Attribute} & \textbf{Abbr} & \textbf{Value}\\\hline
		Country & C & NL\\
		Locality & L & Amsterdam\\
		Organisation & O & Vrije\\
		OrganisationalUnit & OU & Comp.Sc\\
		CommonName & CN & Main Server\\
		Mail\_Servers & -- & 137.37.20.3, 130.37.24.6\\
		FTP\_Server & -- & 130.37.20.20\\
		WWW\_Server & -- & 130.37.20.20	
	\end{tabular}
\end{table}

\subsubsection{Decentralised Schemes}
\begin{itemize}
	\item Driven by advent of peer-to-peer
	\item Need efficient mapping of (attribute,value) pairs to avoid exhaustive search of networks
	\item Two Approachs:
	\begin{itemize}
		\item Distributed Hash Tables (INS/Twine)
		\item Semantic Overlay Networks
	\end{itemize}
\end{itemize}

\begin{note}{Distributed Hash Tables}
	\begin{itemize}
		\item In INS/TWINE each entity (resource) is described by a hierarchical attribute-value tree (AVTree)
		\item Every path from root of AVTree gets unique hash value
		\item Node in DHT responsible for hash value will keep reference to actual resource
		\item Query for (type-book) will get hashed to value 5 and sent to node responsible for storing hash value 5
	\end{itemize}
\end{note}

\begin{note}{Semantic Overlay Networks}
	\begin{itemize}
		\item When there is no organised attribute-based naming resolution scheme, nodes must discover for themselves where resources are located
		\item To make queries efficient, nodes can track nodes with similar resources
		\item Measuring similarity based on attributes is difficult $\rightarrow$ different nodes have different definitions of attributes
		\item Possibly ignore attributes and use file names
		\subitem Similarity measured as number of files in common	
	\end{itemize}
\end{note}

