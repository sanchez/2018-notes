\section{Fault Tolerance}
\subsection{Failure Models}
\begin{description}
	\item[Crash Failure:] server halts, but working correctly until it halts
	\item[Omission failure:] server fails to respond to incoming requests
	\item[Receive omission:] server fails to receive incoming messages
	\item[Send omission:] server fails to send messages
	\item[Timing failure:] server's response outside specified time interval
	\item[Response failure:] server's response incorrect
	\item[Value failure:] value of response wrong
	\item[State transition failure:] server deviates from correct flow of control
	\item[Arbitrary failure:] server may produce arbitrary responses at arbitrary times	
\end{description}

\subsection{Process Resilience}
\subsubsection{Flat vs Hierarchical}
Flat
\begin{itemize}
	\item Symmetry - No 1 point of failure
	\item Complexity - voting	
\end{itemize}
Hierarchical
\begin{itemize}
	\item If coordinator fails, whole thing fails
	\item If coordinator ok, decisions simple	
\end{itemize}

\subsection{Byzantine Generals}
\begin{itemize}
	\item Every process broadcasts it's strength, then all processes broadcast the results they received
	\item For $k$ faulty processes need $2k+1$ correct processes
	\item Total number needed for up to $k$ bad processes: $3k+1$	
\end{itemize}

\subsection{Client Crashes}
\textbf{orphan:} computation with no client\\
Solutions:
\begin{description}
	\item[Extermination:] record RPCs, kill on reboot
	\item[Reincarnation:] broadcast new \textit{epoch} on reboot, all remote computations killed
	\item[Gentle Reincarnation:] only kill if owner not found
	\item[Expiration:] if not given new time quantum, die
\end{description}

\subsection{Scalability in Reliable Multicasting}
Client sends an ACK when it receives a new message. Problem, $N$ receivers, 1 message $\rightarrow N$ ACKs. Only return feedback if message missed

\subsubsection{Non-hierarchical Feedback Control}
If a receivers sees a NACK being sent, suppress sending a NACK.

\subsection{Message Ordering}
Multicasts within epochs defined by view changes (group membership changes). What of ordering within an epoch?
\begin{description}
	\item[unordered:] no guarantee of order processes see messages
	\item[FIFO-ordered:] messages from same process seen in same order
	\item[Causally-ordered:] even if not from same sender causal ordering maintained
	\item[Totally-ordered:] delivered in same order to all group members (in addition to any of the above)
\end{description}

\end{multicols}
\begin{table}[h]
	\centering
	\begin{tabular}{l|l|l}
		\textbf{Multicast} & \textbf{Basic Message Ordering} & \textbf{Total-ordered Delivery?}\\\hline
		Reliable multicast & None & No\\
		FIFO multicast & FIFO-ordered delivery & No\\
		Causal multicast & Causal-ordered delivery & No\\
		Atomic multicast & None & Yes\\
		FIFO atomic multicast & FIFO-ordered delivery & Yes\\
		Causal atomic multicast & Causal-ordered delivery & Yes\\
	\end{tabular}
	
\end{table}
\begin{multicols}{2}

\subsection{Distributed Commit}
\begin{itemize}
	\item Atomic multicasting is an example of a more general problem: \textbf{distributed commit}
	\item Distributed commit requires that an \textbf{operation} is performed by each member of a process group or none at all
	\item Two- and three-phase commit protocols	
\end{itemize}

\subsubsection{Two-Phase Commit}
\begin{description}
	\item[COMMIT:] Make transition to COMMIT
	\item[ABORT:] Make transition to ABORT
	\item[INIT:] Make transition to ABORT
	\item[READY:] Contact another participant
\end{description}
	
\subsubsection{Three-Phase Commit}
The states of the coordinator and each participant satisfy the following two conditions:
\begin{itemize}
	\item There is no single state from which it is possible to make a transition directly to either a COMMIT or an ABORT state
	\item There is no state in which it is \textbf{not} possible to make a final decision, and from which a transition to a COMMIT state can be made	
\end{itemize}

\subsection{Recovery}
\begin{description}
	\item[Backward Recovery:] back from erroneous state to previous correct state
	\item[Forward Recovery:] construct correct state from error state
\end{description}

\subsection{Checkpointing}
Recovery Line
\begin{itemize}
	\item Most recent distributed snapshot
	\item Local states in stable storage used to recover global state	
\end{itemize}
Problem: domino effect from \textit{independent checkpointing}\\
Solution: record dependences to allow consistent state to be found

\subsection{Message-Logging Schemes}
\textbf{Pessimistic Logging}
\begin{itemize}
	\item each nonstable message delievered to at most one process
	\item avoids orphan processes
	\item main work \textit{before} a crash	
\end{itemize}
\textbf{Optimistic Logging}
\begin{itemize}
	\item roll each process back to state where it does not depend on delivery of message which some process(es) do not yet have in stable storage
	\item main work \textit{after} crash	
\end{itemize}
\textbf{Generally}
\begin{itemize}
	\item pessimistic is easier to get right, and more practical	
\end{itemize}



