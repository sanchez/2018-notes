\section{Distributed File Systems}
\subsection{Open Network Computing RPC (ONC RPC)}
\begin{itemize}
	\item hides OS-dependent RPC features
	\item NFS version 4 supports \textit{compound procedures}
	\begin{itemize}
		\item group multiple RPCs
		\item save latency of multiple RPCs
		\item handled in order as if sent separately
		\item if 1 fails, rest terminate
	\end{itemize}	
\end{itemize}

\subsection{Semantics of File Sharing}
\begin{description}
	\item[UNIX semantics:] Every operation on a file is instantly visible to all processes
	\item[Session semantics:] No changes are visible to other processes until the file is closed
	\item[Immutable files:] No updates are possible; simplifies sharing and replication
	\item[Transaction:] All changes occur atomically	
\end{description}

\subsection{File Locking in NFS}
\begin{description}
	\item[Lock:] Creates a lock for a range of bytes
	\item[Lockt:] Test whether a conflicting lock has been granted
	\item[Locku:] Remove a lock from a range of bytes
	\item[Renew:] Renew the lease on a specified lock
\end{description}

\subsection{NFS Summary}
\begin{itemize}
	\item Transparency
	\item No distributed naming policy built in
	\item Performance undone original stateless design
	\item Fault tolerance bit of an afterthought	
\end{itemize}

\subsection{Coda}
\subsubsection{Caching and Replication}
Client-side caching
\begin{itemize}
	\item important for performance
	\item callbacks to maintain consistency
	\begin{itemize}
		\item \textbf{callback promise} - server knows client has file
		\item \textbf{callback break} - after server asked to invalidate copies
	\end{itemize}	
\end{itemize}
Server-side replication
\begin{itemize}
	\item \textbf{optimistic replication} - may be inconsistent
	\item \textbf{versioning} used to make consistent	
\end{itemize}

\subsubsection{Server Replication}
\begin{itemize}
	\item The unit of replication is a \textbf{volume}
	\item \textbf{Volume Storage Group (VSG):} the collection of servers that have a copy of a volume
	\item \textbf{Accessible Volume Storage Group (AVSG):} group of servers in that volume's VSG that the client can contact
	\item Uses a variant of Read-One, Write-All (ROWA) protocol:
	\subitem Read from 1 server
	\subitem Update to all servers in the AVSG	
\end{itemize}

\end{multicols}
\begin{table}[h]
	\centering
	\caption{Summary of NFS vs Code}
	\begin{tabular}{l|l|l}
		\textbf{Issue} & \textbf{NFS} & \textbf{Coda}\\\hline
		Design goals & Access transparency & High availability\\
		Access model & Remote & Up/Download\\
		Communication & RPC & RPC\\
		Client process & Thin/Fat & Fat\\
		Server groups & No & Yes\\
		Mount granularity & Directory & File System\\
		Name space & Per client & Global\\
		File ID scope & File server & Global\\
		Sharing semantics & Session & Transactional\\
		Cache consistency & Write-back & Write-back\\
		Replication & Minimal & ROWA\\
		Fault tolerance & Reliable communication & Replication and caching\\
		Recovery & Client-based & Reintegration
	\end{tabular}
\end{table}
\begin{multicols}{2}
