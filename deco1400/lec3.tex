% !TeX spellcheck = en_US
% !TeX root = notes.tex
\section{Planning and CSS}
\subsection{Determining Content}
\begin{itemize}
	\item Make a list of necessary content (client)
	\item Prioritise the information (user)
\end{itemize}

\subsection{Organisational Schemes}
\begin{itemize}
	\item \textbf{Exact} organisational schemes: Great for known item searching
	\begin{itemize}
		\item Alphabetical
		\item Chronological
		\item Geographical
	\end{itemize}
	\item \textbf{Ambiguous} organisational schemes: Reflects nature of language (imprecise)
	\begin{itemize}
		\item Topical
		\item Task-oriented
		\item User-specific
		\item Metaphor-driven
	\end{itemize}
\end{itemize}

\subsubsection{Hybrid Schemes}
\begin{itemize}
	\item Combines multiple organisational schemes
	\item Unless the schemes are physically separated, confusion will result
	\item They allow for multiple entry points to the content	
\end{itemize}


\subsection{Organisational Structures}
\begin{itemize}
	\item Hierarchy
	\item Hypertext
	\item Database	
\end{itemize}

\subsubsection{Scheme vs Structure}
\begin{itemize}
	\item A \textbf{scheme} groups similar things together
	\item A \textbf{structure} shows how those groups are related	
\end{itemize}

\subsection{Card Sorting}
\subsubsection{Step 1}
Analyse documents for:
\begin{itemize}
	\item Objects (noun)
	\item Actions (verb)
	\item To identify content and tasks for users  on website	
\end{itemize}
\subsubsection{Step 2}
\begin{itemize}
	\item Identify the set of keywords/topics to be categorised
	\item Write each keyword/topic on an index card/post-it note	
\end{itemize}
\subsubsection{Step 3}
Can be done:
\begin{itemize}
	\item In-person with an observer
	\subitem User thinks aloud as they categorise (gives insight into their thought process)
	\item In-person without an observer	
	\subitem Participant works along (can interview after if needed)
\end{itemize}

\subsubsection{Open Card Sort}
\begin{itemize}
	\item Users \textbf{organise} cards into piles (that make sense to them)
	\item Users \textbf{name} each group (in a way that accurately describes the content)
	\item You learn how users group content and what terms/labels they use	
\end{itemize}

\subsubsection{Closed Card Sort}
\begin{itemize}
	\item Provide users with \textbf{pre-defined} categories
	\item Users \textbf{place} cards into a category (that makes most sense to them)
	\item Learn how users relate content to categories
\end{itemize}

