% !TeX spellcheck = en_US
% !TeX root = notes.tex
\section{Navigation Systems}
\subsection{Wayfinding}
\begin{leftbar}
	``How we navigate through complex physical spaces'' - Kevin Lynch: The Image of the City (1960)	
\end{leftbar}
\begin{note}{4 Core Components of Wayfinding}
	\begin{itemize}
		\item Orientation
		\item Route decisions
		\item Mental mapping
		\item Closure	
	\end{itemize}
\end{note}
\begin{note}{5 Common Elements of Wayfinding}
	\begin{itemize}
		\item Paths
		\item Edges
		\item Districts
		\item Nodes
		\item Landmarks
	\end{itemize}
\end{note}

\subsection{Types of Navigation}
\subsubsection{Primary Navigation}
Main header menu
\subsubsection{Secondary Navigation}
Some sub heading under the main menu
\subsubsection{Supplementary Navigation}
Navigation on the side (similar to sort by category)
\subsubsection{Local Navigation}
Navigation of the headings within the current page
\subsubsection{Breadcrumbs}
Provides a hierarchy view of the current page location
\subsubsection{Utility Navigation}
Extra menus and systems hidden away but provided commonly through icons
\subsubsection{Footer Navigation}
Stuff at the bottom of the page
\subsubsection{Global/Universal Navigation}
The whole header banner, included could be:
\begin{itemize}
	\item Primary Navigation
	\item Secondary Navigation
	\item Utility Navigation
\end{itemize}

\begin{note}{What Not To Do!}
	Mystery Meat Navigation
	\begin{itemize}
		\item A visually attractive but inefficient or confusing user interface
		\item Obscures navigation
		\item Forcing user to explore (could be excused)	
	\end{itemize}
	Surprise Dropdowns
	\begin{itemize}
		\item Unclear when dropdowns are available
		\item Always indicate that there is more hidden information	
	\end{itemize}
\end{note}

\begin{note}{Good Practice Standards}
	\begin{itemize}
		\item Use familiar names for links
		\item Clearly distinguish between different types of navigation
		\item Use common positioning	
	\end{itemize}
	
\end{note}

\section{HTML Sectioning and Layout}
\subsection{HTML Sectioning Elements}
\begin{description}
	\item[section:]
	\item[article:]
	\item[aside:]
	\item[nav:]
	\item[header:]
	\item[footer:]	
\end{description}

\section{Visual Organisation}
\subsection{Good Design}
\begin{itemize}
	\item About the relationship between elements
	\item Creating a balance between them
	\item Is timeless (outlasts the fads)
	\item Has a lasting impact on user
	\item Users are pleased by it, but drawn to the content (doesn't get in the way)
	\item Users are able to move easily via the navigation
	\item Creates a cohesive whole	
\end{itemize}

\subsection{Anatomy of a Webpage}
\begin{itemize}
	\item Containing block
	\item Logo/identity/banner
	\item Navigation -- global, secondary
	\item Content -- global, specific to page
	\item White space	
\end{itemize}

\subsection{Layout Principles}
\subsubsection{Proximity}
\begin{itemize}
	\item People perceive items that are located together as being related
	\item Related content should be placed closer together
	\item Unrelated content should be clearly separated
	\item Separate content with white space (empty/negative space)	
\end{itemize}

\subsubsection{Alignment and Positioning}
\begin{itemize}
	\item Concerned with where elements are on a page
	\item Information is easier to digest if in alignment
	\item Positioning elements on a page implies hierarchy/flow	
\end{itemize}

\begin{note}{Balance}
	\begin{itemize}
		\item Elements on the page have ``weight''
		\item Similar to concept of physical balance
		\item Symmetrical
		\item Asymmetrical	
	\end{itemize}
\end{note}

\begin{note}{Patterns}
	\begin{itemize}
		\item Common positions for certain elements
		\item Branding, different types of navigation, calls to action
		\item Across websites
		\item Meet user expectations	
	\end{itemize}
\end{note}

\subsubsection{Emphasis and Contrast}
\textbf{Emphasis} to:
\begin{itemize}
	\item Draw attention to items on the page
	\item Reinforce hierarchy
	\item Use contrast to differentiate elements	
\end{itemize}
\textbf{Contrast} is:
\begin{itemize}
	\item Degree of difference between elements
	\item Very different == high contrast
	\item Not very different == low contrast	
\end{itemize}

\subsubsection{Consistency}
Uniformity and consistency:
\begin{itemize}
	\item In layout of elements
	\item In appearance of elements
	\item Within and across pages
	\item (Doesn't mean you can't have variety)	
\end{itemize}
Each page should appear to belong to the website

\subsection{Wireframes}
\begin{itemize}
	\item Features visual organisation of page anatomy
	\item Usually black and white, sketched appearance, generic
	\item Good for getting feedback from users/clients
	\item It's a technical document!	
\end{itemize}
