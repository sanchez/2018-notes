\section{Prototyping}
\subsection{Evaluation Types}
\begin{description}
	\item[Formative:] Before the release
	\item[Summative:] After completion/the release	
\end{description}

\subsection{When to Evaluate}
\begin{description}
	\item[Early:] From the very start of the project
	\item[Often:] And as often as is practical
\end{description}

\subsection{What to Evaluate}
\begin{itemize}
	\item User Experience
	\item Usability
	\item Content
	\item Functionality
	\item Requirements	
\end{itemize}

\subsection{Who Should Evaluate}
\begin{description}
	\item[Experts:] Domain, usability, design/dev team
	\item[Users:] Experts are not like your users
\end{description}

\subsection{Forms of User Testing}
\begin{itemize}
	\item Moderated vs unmoderated
	\item Scripted vs unscripted
	\item Formal vs informal
	\item And the use of prototypes!	
\end{itemize}

\subsection{What Are Prototypes in Web Design?}
\begin{description}
	\item[Revolutionary Prototypes:] Prototype used briefly then discarded
	\item[Evolutionary Prototypes:] The prototype eventually becomes the product
\end{description}

\subsection{Prototype Resolution}
\subsubsection{Low Fidelity Prototypes}
Low level of detail:
\begin{itemize}
	\item Interactive wireframe
	\item Paper-based prototypes	
\end{itemize}
Deliberately rough to quickly test out specific ideas
\subsubsection{High Fidelity Prototypes}
High level of detail in interface, functionality and task flow. Usually created in final product medium, such as :
\begin{itemize}
	\item HTML/CSS/JS
	\item Mobile app
	\item Final materials	
\end{itemize}

\subsection{Dimensions of Fidelity}
\begin{itemize}
	\item Visual fidelity
	\item Functional fidelity
	\item Content fidelity	
\end{itemize}

\subsection{How are Paper Prototypes Used?}
\begin{description}
	\item[External stakeholders:] Clients and Users
	\item[Internal stakeholders:] Design/dev team, manager, sales
\end{description}

\subsection{Future Design Activities}
\begin{description}
	\item[Paper Prototyping:] Low fidelity functional prototype
	\item[Aesthetics User Testing:] Low fidelity aesthetic prototype
	\item[Hi-Fi User Testing:] High fidelity aesthetic and functional prototype
\end{description}