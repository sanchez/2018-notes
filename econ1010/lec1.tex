% !TeX spellcheck = en_US
% !TeX root = notes.tex
\section{Thinking Like an Economist}
\subsection{What is Economics?}
\begin{leftbar}
	\noindent\textbf{Life} is about making choices.\\
	Economics is the \textbf{science of} choice.\\
	That means economics is the \textbf{science of life.}
\end{leftbar}
by Mr. Alan Duhs (Senior Lecturer, UQ School of Economics
\subsubsection{What is Microeconomics?}
\begin{itemize}
	\item How to use what you have (your resources) to get as much as possible of what you want
	\item It's mostly about how individuals make the most efficient (effective) choices
	\item The systematic effects these choices have on other individuals
\end{itemize}

\begin{note}{Scarcity Principle}
	Our resources are limited, so getting more of one thing means getting less of another.
	\begin{itemize}
		\item Wants exceeds available resources
		\item Choices between alternatives needed
	\end{itemize}
	Something is \textbf{scarce} if you:
	\begin{itemize}
		\item have to sacrifice something else to get it (e.g. money, time, effort)
		\item need to pay a price for it (i.e. not free)
	\end{itemize}
	\begin{description}
		\item[Consumers] will be forced to decide what to consume
		\item[Producers] will be forced to decide what to produce
		\item[Governments] will be forced to decide how to allocate resources to achieve specified objectives
	\end{description}
\end{note}

\subsection{Opportunity Cost}
All about what was \textbf{not} chosen. Economic concept to help make a rational choice. What was sacrificed. What is given up once a decision has been made.

\subsection{Cost Benefit Principle}
Chose to do something only if the \textbf{extra benefit} (incremental benefit) from doing it is greater than (or equal to) the \textbf{extra cost} (incremental cost), assuming the individual is \textbf{rational}.

\subsection{Economic Surplus}
Incremental benefits of an action minus the incremental explicit and implicit costs of that action
\begin{description}
	\item[Explicit cost] a cost that involves spending money (i.e. a transaction physically occurs)
	\item[Implicit cost] a non-monetary \textbf{``opportunity cost''} (no transaction occurs but an alternative is not chosen)
\end{description}
Econmic decision strive to maximize economic surplus by:
\begin{enumerate}
	\item \textbf{maximizing} the benefits
	\item \textbf{minimizing} the costs
\end{enumerate}
Economic surplus can be maximized by making choices that \textbf{minimize the opportunity cost}. \textbf{Opportunity cost} is economics is about assessing if an \textbf{efficient choice} of resources has been made.

\subsection{Rules for Making Rational Economic Choices}
In economics, a rational choice should:
\begin{enumerate}
	\item \textbf{include} opportunity cost
	\item \textbf{exclude} sunk cost
	\item measure cost in \textbf{absolute dollar amount}, not percentages
	\item be based on \textbf{Marginal Analysis}
\end{enumerate}

\begin{note}{Sunk Cost}
	\begin{itemize}
		\item expenses that have occurred in the past before a decision has been taken
		\item costs that would have had to occur in order for a choice to be made
		\item costs that are typically not able to be directly recovered
		\begin{enumerate}
			\item exploration costs (oil well, mining)
			\item market research costs (focus groups, surveys)
			\item feasibility study costs (before a decision is made)
		\end{enumerate}
	\end{itemize}
\end{note}

\subsection{Marginal Benefit}
The change in total benefit from doing \textbf{one extra unit of} an activity
$$ = \frac{\text{change in total benefit}}{\text{one extra unit sold}}$$
\subsection{Marginal Cost}
The change in total cost from doing \textbf{one extra unit of} an activity
$$ = \frac{\text{change in total cost}}{\text{one extra unit produced}}$$

\begin{note}{Economic Efficiency}

\end{note}