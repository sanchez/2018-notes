% !TeX spellcheck = en_US
% !TeX root = notes.tex
\section{Externalities, Common Resources and Property Rights}
\subsection{What are Externalities?}
\begin{note}{Negative Externality (external cost)}
	A cost of an activity that falls on people other than those who pursue the activity.\\
	Impacts from by-products of a production or consumption activity that impose costs on others not directly involved in the production or consumption activity.
\end{note}
When a negative externality is involved:
\begin{enumerate}
	\item $MC_\text{social}=MC_\text{private} + \text{marginal external cost}$
	\item Too much of the good or service will be produced at the private market clearing point, so $MB<\text{marginal social cost}$
	\item By including the costs from negative externalities imposed on society, a socially efficient level of output will be produced, when $MB=\text{Marginal Social Cost}$
\end{enumerate}

\begin{note}{Positive Externality}
	A benefit of an activity received by people other than those who pursue the activity.\\
	A situation where benefits spill over onto others not involved in producing or consuming a product or service. The costs can be reduced or benefits increased, making others better off.
\end{note}
When a positive externality is involved:
\begin{enumerate}
	\item $MB_\text{social}=MB_\text{private} + \text{marginal external benefit}$
	\item \textbf{Not enough} quantity of the good or service will be produced at the private market clearing point, so $MB_\text{social}>\text{marginal cost}$(hence inefficient)
	\item By including the benefits from positive externalities generated for society, a socially efficient level of output will be produced when $\text{Marginal Social Benefit} = MC$
\end{enumerate}

\subsection{Dealing with Externalities}
Competitive markets \textbf{do not} generate an efficient level of output when \textbf{externalities} exist
\begin{itemize}
	\item Negative externalities result in \textbf{more} than the socially efficient level of output being produced
	\item Positive externalities result in \textbf{less} than the socially efficient level of output being produced
\end{itemize}

\subsection{Private remedies}
Agreement is reached between affected parties by themselves.\\\\
How can a more efficient level of production of goods and services in an economy be achieved when externalities do exist?
\begin{enumerate}
	\item Private remedies
	\item Command and control
	\item Taxes and subsidies
	\item Tradable permits
\end{enumerate}
\begin{description}
	\item[Internalise] an externality means that firms and consumers change their behaviour so that an externality is taken into account internally by firms or consumers
	\item[Private remedies] eliminate (or internalise) an externality without a third party (e.g. lecturer, government) having to step in to resolve things
\end{description}
\begin{note}{The Coase Theorem}
	Fundamental to the Theorem is the notion of \textbf{property rights} (the rights for the use, sale and proceeds from a good or resource) which allow effective negations to take place. \textbf{Property rights} help determine:
	\begin{itemize}
		\item who has the right to pollute or infringe on others
		\item who has to pay to make any adjustment needed to rectify the externality
	\end{itemize}
\end{note}

\subsubsection{Efficient solutions to negative externalities can be found if:}
\begin{itemize}
	\item Affected parties can negotiate
	\item No transaction costs involved
\end{itemize}
\textbf{But}, in practice, negotiations are \textbf{not} always practical, so government makes laws to solve the problems of externalities $\rightarrow$ \textbf{Command and Control}

\subsection{Command And Control}
\textbf{Government} imposed \textbf{restrictions or regulations} on individuals or firms to solve the problems from externalities (most laws).
\subsubsection{Disadvantage}
\begin{itemize}
	\item firms have \textbf{no incentive} to find better solutions
	\item new technologies are not encouraged to be found that could be more efficient (do as you are told!)
\end{itemize}

\subsection{Market Based Instruments (MBI, using price)}
MBIs are government \textbf{policies} aimed to positively influence behavior of people in markets to achieve targeted outcomes. Policy examples:
\begin{description}
	\item[Taxes:] (on negative externalities e.g. pollution) discourages activities with negative externalities
	\item[Subsidies:] (on positive externalities e.g. education) encourages activities with positive externalities
\end{description}
\subsubsection{Advantage}
\begin{itemize}
	\item Allows pollution reduction to be achieved by firms that can do so \textbf{at least cost}
\end{itemize}
(recognizes some firms can cut pollutions more cheaply than others)
\subsubsection{Disadvantage}
\begin{itemize}
	\item Difficult to know the tax rate to apply, as governments don't know costs to individual firms to reduce pollution
	\item Sunk costs involved in changing technology
\end{itemize}

\subsection{Tradable permits (``cap and trade'' system)}
Government issues ``pollution permits'' to a desired level (\textbf{cap}) of pollution. Firms can \textbf{trade} permits in a market and pollute only if holding a permit.
\begin{description}
	\item[Intuition:] firms will compare permit cost (allowing it to pollute a specified amount) with the cost of switching to a ``greener technology'' (so as to reduce its pollution by the permit amount so they no longer need the permit).
\end{description}
\begin{itemize}
	\item Establish an upper limit for the quantity of allowable emissions (cap)
	\item Define entitlements and distribute among users
	\item Create a market system to allow trading of entitlements
\end{itemize}
\subsubsection{Advantages over priced based MBIs (taxes):}
\begin{enumerate}
	\item Does not induce firms to invest in ``greener technologies'' (sunk costs) that might be scrapped if government targets not met
	\item Provided incentives for firms to reduce pollution, sell their permit, and profit
	\item Trading of permits allows citizens to purchase permit. Permits might not be sold again by citizens, reducing pollution to lower levels than government originally set
\end{enumerate}