% !TeX spellcheck = en_US
% !TeX root = notes.tex
\section{Public Goods}
\begin{table}[H]
	\centering
	\begin{tabular}{c|cc}
		& Non-rivalrous & Rivalrous\\\hline
		Non-excludable & \textbf{Public} & Common\\
		Excludable & Collective & Private
	\end{tabular}
	\caption{Four Different Types of Goods}
\end{table}
\begin{description}
	\item[Rivalry:] The extent to which \textbf{consumption} of a good (or service) by one person lessens its availability for others. e.g. driving a car
	\item[Non-rivalrous:] The extent to which the consumption of a good (or service) by one person does \textbf{not} lessen its availability for others. e.g. watching 7pm tv news
	\item[Excludability:] The extent to which non-payers can be excluded from consuming a good or service (most of the time!) e.g. buying a new car. If can't \textbf{pay} for it, excluded from getting the car.
	\item[Non-Excludability:] Non-payers are able to consume the good or service. e.g. gaining benefits of national defense even if not paying taxes
\end{description}
\vspace{3em}
\begin{description}
	\item[Public Goods:] Goods and services that have characteristics of being both \textbf{non-rivalrous} and \textbf{non-excludable}
	\item[Private Goods:] Goods and services that have characteristics of being both \textbf{rivalrous} and \textbf{excludable}
	\item[Common Goods:] Goods and services that have characteristics of being both rivalrous and non-excludable
	\item[Collective Goods:] Goods and services that have characteristics of being both \textbf{non-rivalrous} and \textbf{excludable}
\end{description}

\vspace{2em}

\noindent The \textbf{Marginal Benefit} for a \textbf{Public Good}. A conceptual difference exists with a public good (compared to a private good) in that public goods are \textbf{non-excludable} and \textbf{non-rivalrous}. A public good demand is obtained by \textbf{vertically adding} each consumer's willingness to pay.

\subsection{Free-rider problem}
\begin{note}{Free-rider problem}
	The incentive that arises to not contribute to the provision of a good or service in situations in which individuals (or companies, or countries) are able to enjoy the benefits of a good or service without contributing to its cost.
\end{note}
\begin{itemize}
	\item Is why competitive markets will tend to under-provide public goods (as they are not able to cover the costs of production)
	\item Provides the theoretical justification for their possible provision of public goods by the government
\end{itemize}

\subsection{Paying for Public Goods}
\textbf{Not everyone benefits equally} from the provision of a given public good or service.\\
$\rightarrow$ It would seem equitable if people were \textbf{taxed in accordance with their willingness to pay} (it turns out this is efficient as well).\\
$\rightarrow$ Practically though, the government \textbf{lacks precise information} on people's willingness to pay.\\
$\rightarrow$ The design of \textbf{a tax system} can aim to provide a solution to this problem, at least in part.

\subsubsection{Head Tax}
A tax that collects the same amount from every taxpayer no matter what your income is (an equal tax rule)\\
Outcome: This is a form of \textbf{regressive tax}: As income rises, and the tax remains constant, the proportion of tax paid from the income decreases\\
A head tax system rules out of the provision of many worthwhile public goods.

\subsubsection{Proportional Tax}
A specified proportion of income is paid as tax. Rising income means a larger tax is paid to the government.

\subsubsection{Progressive Tax}
As income increases, the proportion of tax paid also increases. Most industrialized countries have some form of a progressive tax system.\\\\

Wealthy people tend to assign greater value to public goods than low-income people (because they have more money).\\
$\rightarrow$ Head tax results in smaller amounts of public goods being available that the wealthy would want and value.\\
$\rightarrow$ Higher income earners being asked to pay more for public goods increases the economic surplus and provides better outcomes for rich and poor.\\
$\rightarrow$ \textbf{Progressive tax} system produces an economically efficient (and effective) outcome.

\subsection{Conclusions}
\begin{enumerate}
	\item As with a private good, the optimal level of provision of a public good is determined by the intersection of the supply and demand curves
	\item The demand curve for a public good represents the \textbf{vertical summation} of consumers' willingness to pay
	\item Competitive markets will tend to under-provide public goods because they are vulnerable to the \textbf{free-rider problem}
	\item The free-rider problem means that public funding is often necessary to provide certain goods and services
	\item A case can be made \textbf{progressive tax systems} are both more equitable and more efficient in the provision of public goods
\end{enumerate}