% !TeX spellcheck = en_US
% !TeX root = notes.tex
\section{Economics of Information}
\begin{note}{Gathering the optimal amount of information}
	Having move information is better than less, but more information is costly to acquire\\
	$\rightarrow$ There can be rising marginal costs associated with collecting information and diminishing returns\\
	$\rightarrow$ Cost-benefit principle indicates a rational consumer will continue gathering information as long as marginal benefit exceeds the marginal cost
\end{note}

\subsection{Expert Advice}
``Experts'' often say that good decision-making means collecting as much information as possible before making a decision. Is this sound advice?

\subsection{Search Risks}
Search often involves both uncertain benefits and costs. In such situations, economists advocate calculating an expected value which is based on probabilities.

\begin{note}{Expected value of a gamble}
	The average outcome you would win (or lose) if you played a particular gamble an infinite number of times.$$=X\times P(X)+Y\times P(Y)+\ldots$$
\end{note}

\begin{description}
	\item[Fair gamble:] a gamble whose expected value is zero
	\item[Better than fair gamble:] a gamble that has an expected value that is positive
	\item[Risk averse person:] someone who would refuse any fair gamble
	\item[Risk neutral person:] someone who would accept any gamble that is fair or better
\end{description}
\newpage
\subsection{Asymmetric Information}
\begin{note}{Definition}
	Situations in which buyers, and sellers, are \textbf{not equally well informed} about the characteristics of goods and services for sale in the market place.
\end{note}
\textbf{The Lemons Model} (by George Akerlof)\\\\
Asymmetric information tends to \textbf{reduce the average quality} of used goods offered for sale.\\
$\rightarrow$ People who have below average cars (lemons) are more likely to want to sell them.\\
$\rightarrow$ Buyers know below average cars are likely to be on the market and lower their reservation price\\\\
Used car prices are low, so people with good cars keep them longer\\
$\rightarrow$ \textbf{the average quality} of used cars \textbf{falls} even further\\
$\rightarrow$ eventually only lemons are for sale\\
$\rightarrow$ \textbf{Information problems reduce economic efficiency in a market}

\subsection{Inefficiencies and Information Issues}
\subsubsection{Principal - Agent Problem}
A situation where an agent, whose actions are \textbf{costly to monitor} and whose objectives are not aligned with those of the principal, takes actions that do not result in the best outcome for the Principal.
\begin{note}{Example}
	Shareholders are owners of a firm (principles) while managers run the firm (agents).\\\\
	\textbf{Principles} want to maximize profits. \textbf{Agents} would like to maximize their salaries, enjoy spacious well furnished offices etc.
\end{note}

\subsubsection{Adverse Selection Problem}
Those on the informed side of the market self select in a way that tends to reduce the average quality of the good or service sold.

\subsubsection{Moral Hazard Problem}
The tendency of people to change their behavior once they become party to a contract

\subsection{The Costly to Fake Principle}
The idea that to communicate information credibly to a potential rival, a signal must be given that is costly, or difficult, to fake.