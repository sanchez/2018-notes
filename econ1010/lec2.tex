% !TeX spellcheck = en_US
% !TeX root = notes.tex
%\section{Thinking Like An Economist 2}
\subsection{Absolute and Comparative Advantage}
\subsubsection{Absolute Advantage}
\begin{itemize}
	\item ability of an individual, firm, or country \textbf{to produce more} of a product or service than competitors using the \textbf{same amount} of resources.
	\item alternatively, produce the \textbf{same} amount of product or services as competitors with \textit{less resources}.
\end{itemize}
\subsubsection{Comparative Advantage}
\begin{itemize}
	\item ability of an individual, firm, or country to produce a product or service at a \textit{lower opportunity cost} than other competitors (relates to who is more efficient at producing something).
\end{itemize}
Opportunity cost is about assessing if an \textbf{efficient choice} of resources has been made. Outcomes are efficient if opportunity cost is minimised. \textbf{Comparative advantage} exists with the producer (or service provider) producing the product at the \textbf{lowest opportunity cost}. Contrast \textbf{absolute advantage} which is \textit{irrelevant} in deciding who is more efficient at producing something.

\subsection{Gains and Specialization}
\begin{note}{Principle of Comparative Advantage}
	\begin{itemize}
		\item Everyone does best (individuals or countries) when they concentrate on activities for which their opportunity cost is lowest.
		\item By exchanging goods with others, individuals can more efficiently obtained their preferred mix of goods and services.
	\end{itemize}
\end{note}

\subsection{Production Possibility Curve (PPC)}
\begin{itemize}
	\item The production possibilities curve (PPC) = a graphical representation describing the maximum amount of one good that can be produced for every possible level of production of another good.
	\item\textbf{Assumptions:}
	\begin{enumerate}
		\item only two goods are able to be produced (for simplification), bananas and blueberries
		\item consider the PPC for a single worker only
	\end{enumerate}
\end{itemize}
\begin{description}
	\item[Attainable Point:] Any combination of goods that can be produced using currently available resources. All points on the PPC, as well as below and to the left of the PPC, are attainable.
	\item[Unattainable Point:] Any combination of goods that cannot be produced using currently available resources. All points lying above and to the right of the PPC are unattainable.
	\item[Efficient Point:] Any combination of goods for which currently available resources \textbf{do not} allow an increase in the production of one good unless there is a reduction in the production of the other.
	\item[Inefficient Point:] Any combination of goods for which currently available resources \textbf{enable} an increase in the production of one good \textbf{without} a reduction in the production of the other.
\end{description}