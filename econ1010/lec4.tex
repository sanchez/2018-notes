% !TeX spellcheck = en_US
% !TeX root = notes.tex
\begin{note}{Increase in Demand}
	\textbf{Increase} in demand $\rightarrow$ \textbf{Right shift} in demand $\rightarrow$ \textbf{Positive movement} in demand\\Figure~\ref{fig:left} shows the curve shifting.
\end{note}
\begin{figure}[H]
	\centering
	\begin{tikzpicture}
		\begin{axis}[axis lines=left,xlabel={Price (\$/can)},ylabel={Quantity ('000s of cans)}]
			\addplot[domain=0:16,color=blue]{4-(x/4)} node[above,pos=0.3] {$D_1$};
			\addplot[domain=0:24,color=red]{6-(x/4)} node[above,pos=0.3] {$D_2$};
			\node[dot] (left) at (axis cs:8,2){};
			\node[dot] (right) at (axis cs:16,2){};
			\draw[->, line width=0.7mm] (left) -- (right);
			\draw[dotted, line width=0.5mm] (axis cs:0,2) -- (left);
			\draw[dotted, line width=0.5mm] (axis cs:8,0) -- (left);
			\draw[dotted, line width=0.5mm] (axis cs:16,0) -- (right);
		\end{axis}
	\end{tikzpicture}
	\caption{Right Shift in demand}\label{fig:right}
\end{figure}
\begin{note}{Decrease in Demand}
	\textbf{Decrease} in demand $\rightarrow$ \textbf{Left shift} in demand $\rightarrow$ \textbf{Negative movement} in demand\\Figure~\ref{fig:right} shows the curve shifting.
\end{note}
\begin{figure}[H]
	\centering
	\begin{tikzpicture}
		\begin{axis}[axis lines=left,xlabel={Price (\$/can)},ylabel={Quantity ('000s of cans)}]
			\addplot[domain=0:16,color=blue]{4-(x/4)} node[above,pos=0.3] {$D_1$};
			\addplot[domain=0:24,color=red]{6-(x/4)} node[above,pos=0.3] {$D_2$};
			\node[dot] (left) at (axis cs:8,2){};
			\node[dot] (right) at (axis cs:16,2){};
			\draw[->, line width=0.7mm] (right) -- (left);
			\draw[dotted, line width=0.5mm] (axis cs:0,2) -- (left);
			\draw[dotted, line width=0.5mm] (axis cs:8,0) -- (left);
			\draw[dotted, line width=0.5mm] (axis cs:16,0) -- (right);
		\end{axis}
	\end{tikzpicture}
	\caption{Left Shift in demand}\label{fig:left}
\end{figure}

\section{Factor Causing a Shift in Demand}
\begin{itemize}
	\item Change in consumer taste (or preference)
	\item Change in population
	\item Change in expectations of future price rises
	\item Change price of a substitute
	\item Change in price of a complement
	\item Change in income (and the product is normal)
	\item Change in income (and the product is inferior)
\end{itemize}
Requires the assumption that \textbf{price remains fixed} and some other factor changes so as to affect the quantity demanded by the consumer
\begin{note}{Substitutes}
	A product or service that \textbf{can be used in place of} other products or services. i.e. used as replacements.
\end{note}
\begin{note}{Complements}
	Products or services that \textbf{are consumed together}.
\end{note}
\begin{note}{Normal Goods}
	As \textbf{income increases}, consumer demand \textbf{increases} for the product or service. As \textbf{income decreases}, consumer demand \textbf{decreases} for the product or service
\end{note}
\begin{note}{Inferior Goods}
	As \textbf{income decreases}, consumer demand \textbf{increases} for the product or service. As \textbf{income increases}, consumer demand \textbf{decreases} for the product or service
\end{note}

\section{Factor Causing a Shift in Supply}
\begin{itemize}
	\item Change in number of suppliers in the market
	\item Change in expectations of future selling price
	\item Change in level of input costs
	\item Change production to a new, substitute product
	\item Changes in technology
	\item Changes in government taxes on a product
\end{itemize}
Again, a shift assumes \textbf{price remains fixed} and some other factor changes to affect the quantity supplied by the producer
\subsection{A Shift in Supply}
\begin{align*}
\text{Profit} &= \text{revenue} - \text{expenses}\\
&= \frac{\text{price}}{\text{unit}}\times\text{quantity}-\text{expenses}
\end{align*}