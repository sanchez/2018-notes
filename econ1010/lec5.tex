% !TeX spellcheck = en_US
% !TeX root = notes.tex
\section{Elasticity}
\begin{note}{Elasticity Definition}
	An economic term used to measure the \textbf{sensitivity} of one variable (e.g. quantity demanded) to a change in another variable (e.g. price).\\
	$e_d$: Price elasticity of demand\\
	$e_s$: Price elasticity of supply\\
	\begin{itemize}
		\item The percentage change in quantity demanded divided by the percentage change in price
		$$ e_d = \frac{\%\Delta Q_d}{\%\Delta P} $$
		\item Is a number less than 1, equal to 1, or larger than 1
		\item Is a dimensionless number (no units)
	\end{itemize}
\end{note}
\begin{itemize}
	\item $e_d$ will be a negative number since ``demand'' is a downward sloping relationship
	\item For convenience we \textbf{drop the negative sign} and use the absolute value when talking about elasticity of demand, $e_d$
\end{itemize}
$$ e_d = \frac{P_i}{Q_i}\times\frac{1}{\text{slope of demand curve}} $$

\subsection{Point Price Elasticity of Demand}
\begin{enumerate}
	\item Measures the percentage change in quantity demanded in response to a one percent change in price
	\item Varies depending on the \textbf{price} at a point on the demand curve ($P_i$)
	\item Varies depending on the \textbf{quantity} at a point on the demand curve ($Q_i$)
	\item Varies depending on the \textbf{slope} at a point on the demand curve
\end{enumerate}
\begin{enumerate}
	\item \textbf{Inelastic} if $e_d < 1$
	\item \textbf{Unit Elastic} if $e_d = 1$
	\item \textbf{Elastic} if $e_d > 1$
\end{enumerate}

\subsection{Calculating Midpoint Elasticity}
How can you realistically find the slope of the demand curve at a particular price and quantity demanded? Is demand really linear? An approximation to the \textbf{point price elasticity} is called the \textbf{midpoint price elasticity}.
\begin{align*}
e_d &= \frac{\Delta Q/Q_\text{Average}}{\Delta P/P_\text{Average}}\\
&= \frac{\Delta Q/\left(\frac{Q_A + Q_B}{2}\right)}{\Delta P/\left(\frac{P_A + P_B}{2}\right)}\\
&= \frac{\Delta Q/\left(Q_A + Q_A\right)}{\Delta P/\left(P_A + P_B\right)}
\end{align*}

\subsection{Summary}
\begin{itemize}
	\item If demand is \textbf{inelastic}, price increases, revenue increases
	\item If demand is \textbf{unit elastic}, revenue is maximized
	\item If demand is \textbf{elastic}, price increases, revenue decreases
	\item Elasticity is NOT constant along the demand curve
	\item As price rises, demand becomes \textbf{relatively more elastic} (or referred to as becoming less inelastic)
\end{itemize}
\subsubsection{Factors affecting elasticity of demand}
\begin{enumerate}
	\item Availability of close substitutes
	\begin{itemize}
		\item The more substitutes available for a product, the easier it is for consumers to switch when there is a price increase
		\item Ability of consumers to switch easily implies there is intense competition
		\item The \textbf{more competition}, the \textbf{more elastic} will be demand
	\end{itemize}
	\item Time involved from the time of the price change
	\begin{itemize}
		\item When price change, consumers take time to adjust their spending habits
		\item The more time that passes after a price rise, the more options consumers might be able to find to switch to another product
		\item the \textbf{more time involved}, the \textbf{more elastic} will be demand
	\end{itemize}
	\item Essentials verses luxuries
	\begin{itemize}
		\item The more \textbf{essential} it is for consumers to have an item, the \textbf{less likely they are to switch} to another product (e.g. razor blades for shaving)
		\item A \textbf{luxury} item is non-essential, and so consumers are more easily able to \textbf{choose note to purchase} it (e.g. tickets to a day at the cricket)
		\item The \textbf{more essential a product}, the \textbf{more inelastic} will be demand
	\end{itemize}
	\item The proportion of a consumer's budget spent on the product
	\begin{itemize}
		\item The larger the proportion spent on an item, the more careful a consumer will be before making the purchase. The smaller the proportion, the less worried they will be about the purchase
		\item The \textbf{more} a product takes from a consumer's budget, the \textbf{more elastic} will be demand
	\end{itemize}
\end{enumerate}

\subsection{Price Elasticity of Supply}
\begin{enumerate}
	\item It varies depending on the \textbf{price} at a point on the supply curve ($P_i$)
	\item It varies depending on the \textbf{quantity} at a point on the supply curve ($Q_i$)
	\item It varies depending on the \textbf{slope} at a point on the supply curve
\end{enumerate}