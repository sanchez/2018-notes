% !TeX spellcheck = en_US
% !TeX root = notes.tex
\section{Perfectly Competitive Supply}
An understanding of production \textbf{costs} and assessing their impact in making decisions. Using \textbf{marginal analysis} to help make decisions to \textbf{maximise profits} in \textbf{competitive markets}.
\subsection{Perfectly Competitive Market Features}
\begin{enumerate}
	\item\textbf{Many buyers and many sellers} (small in size). No one buyer or seller is able to dictate price
	\item\textbf{Easy entry and exit into the industry} (no barriers or large exit costs)
	\item\textbf{Products are homogeneous} (identical with no ability to differentiate, all perfect substitutes)
	\item\textbf{Resources are perfectly mobile} (no transport costs, able to freely enter and leave)
	\item\textbf{Perfect knowledge and information} (both buyers and sellers) about others in the industry
\end{enumerate}
\begin{note}{Short Run}
	That time period when \textbf{some inputs are fixed (e.g. capital)} and \textbf{some inputs can change}
\end{note}
\begin{note}{Long Run}
	That time period when \textbf{all inputs can change}, including technology and physical size of the operations
\end{note}
\begin{note}{Total Fixed Costs (TFC, in \$)}
	Costs that remain constant as output either increases or decreases
\end{note}
\begin{note}{Total Variable Costs (TVC, in \$)}
	Costs that vary as the output increases or decreases. e.g. rent, insurance
\end{note}
\begin{note}{Total Costs (TC, in \$)}
	Includes all the costs a firm uses in its production. $$ TC = TFC + TVC $$
\end{note}
\begin{note}{Average Total Costs (ATC, in \$/unit)}
	The total cost of producing output quantity Q (units), divided by the total output quantity. $$ ATC = \frac{TC}{Q}$$
\end{note}
\begin{note}{Average Variable Costs (AVC, in \$/unit)}
	Total variable costs of producing \textbf{output quantity Q (units)}, divided by the total output quantity. $$ AVC = \frac{TVC}{Q} $$
\end{note}
\begin{note}{Marginal Cost (MC, in \$/unit)}
	The \textbf{change in total cost} to a firm to \textbf{produce one more unit} of a good or service. Note the units are \textbf{\$/unit}, and one extra unit of output can be a very large number. $$ MC = \frac{\Delta TC}{\Delta Q} $$
\end{note}
\subsubsection{Costs are one thing, but what about Profit?}
\begin{enumerate}
	\item Profit = Revenue - Expenses
	\item Revenue = price * quantity
	\item Price depends on many things (elasticity of demand, differentiation, substitutes etc..., and the type of \textbf{Market Structure})
	\item For simplicity at this stage, consider a \textbf{Perfectly Competitive Market}
\end{enumerate}

\subsection{Loss Minimising Analysis}
Minimise any losses when $AVC < P < ATC$ by producing an output at the point where $P=MC$. All of the variable costs will able to be paid under these conditions, and any losses will be minimized. Only part of the fixed costs will able to be paid, not all. Will a business be able to survive? For how long?