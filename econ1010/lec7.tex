% !TeX spellcheck = en_US
% !TeX root = notes.tex
\section{Quest for profit and the Invisible Hand}
Three different definitions of profit:
\begin{itemize}
	\item Accounting Profit
	\item Economic Profit
	\item Normal Profit
\end{itemize}
\begin{note}{Explicit Costs}
	The actual payments (or transactions) a firm makes to other suppliers for the factors of production it needs
\end{note}
\begin{note}{Implicit Costs}
	The next best alternative sacrificed when the firm decided to use its resources in the chosen industry it operates in. i.e. implicit cost is an \textbf{opportunity cost}
\end{note}
\begin{align*}
	\text{Account Profit} &= \text{Revenue} - \text{Total Cost}\\
	&= \text{Revenue} - \text{explicit cost}\\
	\text{Economic Profit} &= \text{Revenue} - \text{Total Cost}\\
	&= \text{Revenue} - (\text{explicit cost} + \text{implicit costs})\\
	&= (\text{Revenue} - \text{explicit cost}) - \text{implicit costs}\\
	&= (\text{Accounting profit}) - \text{implicit costs}
\end{align*}
If economic profit is a ``positive number'', the profit is called \textit{excess profit}, \textit{super normal profit}, or \textit{pure profit}.
\begin{note}{Economic Profit}
	Determins the owner's incentive to keep their resource in the current industry or exit and use their resources more efficiently elsewhere.\\
	If \textbf{economic profit} is \textbf{positive}, this will act as an incentive for owners to \textbf{remain} in the industry, but it will attract new entrants.\\
	If \textbf{economic profit} is \textbf{negative}, this will act as an incentive for owners to \textbf{exit} the industry and deter new entrants from entering.
\end{note}
When economic profit is 0, this is referred to as ``normal profit''.
\begin{note}{Normal Profit}
	\textbf{Normal profit} indicates the opportunity cost of resources (cost of capital) has been covered by revenues generated by business operations where the resources were used. No incentive (attraction) for new competitors to enter, or currently operating firms to leave.
\end{note}

\subsection{Adam Smith's Invisible Hand Theory}
Market price in free market system serves two functions:
\begin{description}
	\item[Rationing function:] distributes scare goods to consumers who value them the most
	\item[Allocative function:] directs resources away from overcrowded markets and towards underserved markets
\end{description}
\begin{leftbar}
	Actions of self interested buyers and sellers, all acting independently, will often result in the socially optimal allocation of resources.
\end{leftbar}
Smith believed \textbf{economic} profit and \textbf{economic} loss are the only forces needed to drive the system of efficient allocation of resources
\begin{itemize}
	\item If economic \textbf{profit is positive}:
	\begin{itemize}
		\item incentive for existing owners to stay in the industry
		\item incentive for new owners to enter into the industry
	\end{itemize}
	\item If economic \textbf{profit = 0 (normal profit)}:
	\begin{itemize}
		\item no incentive for existing owners to exit or new owners to enter the industry
	\end{itemize}
	\item If economic \textbf{profit is negative}:
	\begin{itemize}
		\item acts as an incentive for existing owners to exit the industry and cause others to not want to enter
	\end{itemize}
\end{itemize}