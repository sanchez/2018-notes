% !TeX spellcheck = en_US
% !TeX root = notes.tex
Imperfect competition is when a firm has at least some ability to dictate the selling price (demand is downward sloping). This is the same as saying a firm has some market power (pricing power). Imperfectly competitive markets include monopoly, monopolistic, and \textbf{oligopoly} firms.

\begin{note}{Price Discrimination}
	Occurs when the \textbf{same item} is sold at \textbf{different prices} to \textbf{different groups} of people, depending on customer's willingness to pay
\end{note}

\subsection{Different Types of Price Discrimination}
\subsubsection{First Degree (perfectly) Discrimination}
\begin{itemize}
	\item The firm must know the maximum amount each and every customer will pay for their product or service
	\item Each customer can be charged a different price for the same product and the product will be sold to them
	\item The full consumer surplus is extracted from each customer, so firm maximizes profit
\end{itemize}

\subsubsection{Second Degree Discrimination}
\begin{itemize}
	\item A firm takes part of the consumer surplus (but not all)
	\item Different prices are charged for different ``blocks'' of quantities consumed
	\item Results in larger revenues and profits compared to charging a single lower price for larger quantities
\end{itemize}

\subsubsection{Third Degree Discrimination}
\begin{itemize}
	\item Different \textbf{market segments} charged different prices because of \textbf{differences in price elasticity} of demand
	\item Economic models can be used to show how a firm should operate to maximize profits
\end{itemize}

\subsubsection{Peak Load Pricing}
\begin{itemize}
	\item Suppliers face peak (maximum) demands at particular times (hourly, daily, weekly, yearly)
	\item Pricing based on efficiency measures and reflects costs of supply (i.e. marginal cost)
	\item Typically MC will be higher for suppliers in peak period times (because of capacity restraints for production). Prices are higher in peak periods
\end{itemize}

\begin{note}{Why use it?}
	A firm's profit will be larger by using price discriminating compared to charging all buyers the same price. Firms can aim to extract as much expenditure from consumers as possible by understanding how much they are willing to pay. Theoretically, a firm aims to exert market power to \textbf{capture} as much \textbf{consumer surplus} as possible and convert it into \textbf{producer surplus}.
\end{note}
Comments:
\begin{itemize}
	\item is not necessarily a bad thing
	\item results in an increase in the level of output compared to if a single price is charged
	\item in increasing output toward the perfectly competitive output level, price discrimination can be thought of as a ``good thing'' in terms of enhancing economic efficiency
\end{itemize}

\subsection{Oligopoly Market}
A market structure where a small number of \textbf{interdependent} firms compete (game theory).\\ Key Features include:
\begin{enumerate}
	\item Firms are large, posses a larger market share (say 80\%), and a dominate production
	\item Large initial capital investment needed to start operations (barriers to entry high)
	\item Price is set above marginal cost for the profit maximizing condition
	\item Typically there are only a few substitutes available
	\item Products are slightly differentiated
	\item Able to make significant economic profits
\end{enumerate}
\subsubsection{Why an Oligopoly Market can exit}
Existence primarily from barriers to entry
\begin{enumerate}
	\item Large capital investment needed to enter
	\item Economies of scale exits (which pushes ATC down as output increases so others are unable to compete)
	\item Government imposed restrictions on firms from entering through patents, licensing agreements, tariffs and quotas on foreign competition
\end{enumerate}
\begin{note}{Strategic \textbf{Interdependence} of Oligopoly Firms}
	The \textbf{actions of one firm} can have major impacts on \textbf{actions of other firms} in the same market
\end{note}
\subsubsection{Market Equilibrium in Oligopoly Markets}
For perfect competition, monopolistic, and monopoly markets, \textbf{profit is maximized} then $MR=MC$. For Oligopoly markets, there is \textbf{no simple profit maximizing rule!} Generally, equilibrium is said to exist when a firm is doing the best is can, given what other firms are doing, and when it has no reason to change price or inputs. Interactions (and firm interdependence) is analysed using \textbf{game theory}. ``Business is all about the game!'' Game theory characteristics:
\begin{itemize}
	\item \textbf{Players}
	\item \textbf{Strategies} (possible actions players can choose)
	\item \textbf{Payoffs} (outcomes from strategies)
\end{itemize}

\subsection{Game Theory Analysis}
\begin{description}
	\item[Dominant Strategy:] a strategy for a firm that gives it a \textit{higher payoff} no matter what strategy the other players in a game chose. That is, no matter what a competitor does, the firm will always chose the same strategy
	\item[Dominated Strategy:] any other strategy available to a player who has a dominant strategy (where the dominant strategy leads to a \textit{lower payoff} than if the dominated strategy was chosen)
\end{description}
\subsubsection{Prisoner's Dilemma Game Theory Analysis}
\begin{enumerate}
	\item It is a game where each player has a dominant strategy, and when each plays their dominant strategy, the resulting payoffs are smaller than if each had played a \textbf{dominated} strategy
	\item It is an example of a \textbf{Nash Equilibrium} when each player's adopted strategy is the best that can be chosen, given the other player's chosen strategies
\end{enumerate}
\subsubsection{Prisoner's Dilemma and Imperfectly Competitive Firms}
\textbf{Collusion}
\begin{itemize}
	\item An agreement between firms to charge the same price (price fixing) to avoid competing
	\item Formation of Cartels (oil cartel OPEC)
	\item Illegal in Australia (ACCC watching)
\end{itemize}
\textbf{Cartels} explicitly agree to cooperate in price. Aim is to earn an economic profit. Characteristics needed for product/service:
\begin{enumerate}
	\item Inelastic demand
	\item Monopoly pricing power
	\item Cartel supplies a large percentage of product to the entire market
	\item Cartel members must maintain the agreed high price and NOT cheat! Otherwise price falls
\end{enumerate}
\subsubsection{Repeated Prisoner's Dilemma Game}
A game where the same two players play a prisoner's dilemma over and over many times. Outcomes from all previous plays are observed before the next play begins
\subsubsection{Tit-for-tat repeated prisoner's dilemma game}
A strategy for playing the repeated prisoner's dilemma game in which a player cooperates on the first move, then mimics the other player's last move on each successive moves
\subsubsection{Credible threat}
\begin{itemize}
	\item A threat to take action that is in the threatener's interest to carry out
	\item It is one that can be believed could happen as that player has the ability to carry through on the threat
\end{itemize}
\subsubsection{Credible promise}
\begin{itemize}
	\item A promise to take action that is in the promiser's interests to keep, when the time comes to deliver
\end{itemize}
\subsubsection{Games where Timing matters}
So far we have considered competitive and cooperative \textbf{simultaneous} games. However, many games also include a \textbf{time dimension} where on player gets to go first. These games can be analyzed using a diagram called a \textbf{decision tree}. The decision tree describes the possible moves players can make and shows the sequence of \textbf{possible moves over time}. The tree also lists the payoffs that correspond to each possible combination of moves.