% !TeX spellcheck = en_US
% !TeX root = notes.tex
\chapter{Big Old Summary}

``Ceteris paribus'' means all else being equal.

\begin{note}{Scarcity Principle}
	Our resources are limited, so getting more of one thing means getting less of another. Something is \textbf{scarce} if you:
	\begin{itemize}
		\item have to sacrifice something else to get it
		\item need to pay a price for it
	\end{itemize}
\end{note}

\begin{description}
	\item[Opportunity Cost:] What was not chosen
	\item[Cost Benefit Principle:] Do something only if the \textbf{marginal benefit} is greater than the \textbf{marginal cost}
	\item[Explicit Cost:] Cost involve spending money
	\item[Implicit Cost:] a non-monetary \textbf{``opportunity cost''}
	\item[Sunk Cost:] expenses that have occurred in the past
	\item[Marginal Benefit:] Change in benefit for doing \textbf{one extra unit} of an activity
	\item[Marginal Cost:] Change in cost from doing \textbf{one extra unit} of an activity
	\item[Absolute Advantage:] Produce more of a product, using less resources
	\item[Comparative Advantage:] Produce more at a lower opportunity cost (irrelevant in deciding who is more efficient)
	\item[Competitive Market:] Many buyers and sellers, clearing point is reached
	\item[Price Floor:] Price can be set higher than market clearing
	\item[Price Ceiling:] Price can be set lower than market clearing
	\item[Market Failure:] an inefficient allocation of goods and services in a market
	\item[Producer Surplus:] Market clearing price less the minimum price a supplier would have been willing to accept in a sale
	\item[Economic Surplus:] Total consumer surplus plus total producer surplus
	\item[Dead Weight Loss:] Economic inefficiency
	\item[Substitute:] Used as a replacement to the good
	\item[Complements:] Used with the good
	\item[Normal Goods:] As \textbf{income increases}, consumer \textbf{demand increases}
	\item[Inferior Goods:] As \textbf{income decreases}, consumer \textbf{demand decreases}
	\item[Elasticity:] Measures the \textbf{sensitivity} of one variable (e.g. quantity demanded) to a change in another variable (e.g. price)
	\item[Oligopoly Market:] A market structure where a small number of \textbf{inter-dependent} firms compete (game theory)
	\item[Dominant Strategy:] Strategy firm picks no matter what the other firms do
	\item[Dominated Strategy:] Any other strategy available to a player who has a dominant strategy
	\item[Negative Externality:] Cost of an activity that falls on other people
	\item[Positive Externality:] Benefit of an activity that falls on other people
	\item[Command and Control:] \textbf{Government} imposed \textbf{restrictions or regulations} on firms, solves externalities problem
	\item[Market Based Instruments (MBI):] Policies aimed to positively influence behavior. (Taxes for negative and subsidies for positive)
	\item[Tradable permits (``cap and trade'' system):] Government issue permits at a desired level. Firms \textbf{trade} permits.
	\item[Free-rider problem:] Incentive to not contribute to a good, in which you can enjoy the good without paying
	\item[Head Tax:] Tax that collects the same amount from everyone
	\item[Proportional Tax:] A portion of income is paid as tax
	\item[Progressive Tax:] As income increases, proportion of tax paid also increases
	\item[Fair gamble:] A gamble whose expected value is zero
	\item[Better than fair gamble:] A gamble that has an expected value that is positive
	\item[Risk averse person:] Someone who would refuse an fair gamble
	\item[Risk neutral person:] Someone who would accept any gamble that is fair or better
	\item[Asymmetric Information:] Situations in which buyers, and sellers, are \textbf{note equally well informed} about goods on the market
	\item[Principal - Agent Problem:] Agents (who are costly to monitor) have objectives that don't line up with the principles (owners of the firm)
	\item[Adverse Selection Problem:] Informed people self select which tends to reduce average quality of good
	\item[Moral Hazard Problem:] Tendency of people to change their behavior once they become party to a contract
\end{description}

\begin{note}{Production Possibility Curve (PPC)}
	\begin{description}
		\item[Attainable Point:] Can be achieved with any combination of goods
		\item[Unattainable Point:] Can not be achieved with any combination of goods
		\item[Efficient Point:] A point that lies on the line of the PPC
		\item[Inefficient Point:] Goods can be increased and won't result in another good being decreased
	\end{description}
\end{note}

Supply and Demand, if you don't know what this is there is no hope for you.\\

A \textbf{change in quantity demanded} results in a shift \textbf{along} the demand curve. A \textbf{change in the demand} results in a shift \textbf{of the} demand curve. A \textbf{change in the quantity supplied} results in a shift \textbf{along} the supply curve. A \textbf{change in supply} results in a shift \textbf{of the} supply curve.

\section{Elasticity}
$e_d$: Price elasticity of demand\\
$e_s$: Price elasticity of supply\\
\begin{align*}
	e_d &= \frac{\%\Delta Q_d}{\%\Delta P}\\
	&= \frac{\Delta Q/(Q_A+Q_B)}{\Delta P/(P_A+P_B)}
\end{align*}
\begin{itemize}
	\item If demand is \textbf{inelastic}, price increases, revenue increases
	\item If demand is \textbf{unit elastic}, revenue is maximized
	\item If demand is \textbf{elastic}, price increases, revenue decreases
	\item Elasticity is NOT constant along the demand curve
	\item As price rises, demand becomes \textbf{relatively more elastic} (or referred to as becoming less inelastic)
\end{itemize}

\section{Perfectly Competitive Market}
\begin{enumerate}
	\item\textbf{Many buyers and many sellers} (small in size). No one buyer or seller is able to dictate price
	\item\textbf{Easy entry and exist into the industry} (no barriers or large exit costs)
	\item\textbf{Products are homogeneous} (identical with no ability to differentiate, all perfect substitutes)
	\item\textbf{Resources are perfectly mobile} (no transport costs, able to freely enter and leave)
	\item\textbf{Perfect knowledge and information} (both buyers and sellers) about others in the industry
\end{enumerate}
\begin{description}
	\item[Short Run:] That time period when \textbf{some inputs are fixed} (e.g. capital) and \textbf{some inputs can change}
	\item[Long Run:] That time period when \textbf{all inputs can change}, including technology and physical size of the operations
	\item[Total Fixed Costs (TFC):] Costs remain same regardless of output
	\item[Total Variable Costs (TVC):] Costs vary based on output
	\item[Total Costs (TC):] $TC = TFC + TVC$
	\item[Average Total Costs (ATC):] $ATC=\frac{TC}{Q}$
	\item[Average Variable Costs (AVC):] $AVC=\frac{TVC}{Q}$
	\item[Marginal Cost (MC):] $MC=\frac{\Delta TC}{\Delta Q}$
\end{description}

$$\text{Economic Profit} = \text{Revenue} - EC - IC$$
$EC$: Explicit Cost\\
$IC$: Implicit Cost
\begin{leftbar}
	If economic profit is a ``positive number'', the profit is called \textit{excess profit, super normal profit,} or \textit{pure profit}.
\end{leftbar}

\begin{description}
	\item[Rationing function:] Distributes scare goods to consumers who value them the most
	\item[Allocative function:] directs resources away from overcrowded markets and towards under-served markets
\end{description}

\begin{note}{Perfectly Competitive Firm}
	Demand is perfectly elastic (horizontal), is a \textbf{price taker}, maximizing profit where quantity exists so that $P=MC$
\end{note}
\newpage
\begin{note}{Imperfectly Competitive Firm}
	Faces a downward sloping demand curve, is a \textbf{price maker}, maximizing profit at a specific output quantity (\textit{not} just selling whatever it likes and charging any price)
\end{note}

\section{Monopoly}
A firm that is the \textbf{only} seller of a product or service. It has no \textbf{close} substitutes. The monopoly (firm) supply is the \textbf{market supply}.
\begin{align*}
	\text{Marginal Revenue} &= \text{Marginal Cost}\\
	&= \frac{\Delta\text{Revenue}}{\Delta\text{Quantity}}
\end{align*}
A firm that profit maximizes, at an output quantity where \textbf{price} is much larger than \textbf{MC}, is said to have \textbf{Market Power}. For a single supplier, the higher the price is able to be set above \textbf{MC}, the more \textbf{monopoly power}.\\

Australian and Consumer Competition Commission (ACCC)

\subsection{Monopolistic Competition}
A larger number of similar firms exist. Many close substitutes exist. Few barriers to entry into the industry. Products/services can be slightly \textbf{differentiated} giving the firm some pricing power.

\section{Price Discrimination}
Same item is sold at different prices to different groups of people.
\begin{description}
	\item[First Degree (perfectly) Discrimination:] Firm knows the maximum willingness to pay and charges that
	\item[Second Degree Discrimination:] Different prices are charged for different ``blocks'' of quantities consumed
	\item[Third Degree Discrimination:] Different \textbf{market segments} charged different prices because of \textbf{differences in price elasticity} of demand
	\item[Peak Load Pricing:] Suppliers face peak (maximum) demands at particular time
\end{description}

\begin{note}{Dealing with Externalities}
	\textbf{Negative externalities} result in \textbf{more} than the socially efficient level of output being produced. \textbf{Positive externalities} result in \textbf{less} than the socially efficient level of output being produced.
\end{note}

\section{Public Goods}
\begin{table}[H]
	\centering
	\begin{tabular}{c|cc}
		& Non-rivalrous & Rivalrous\\\hline
		Non-excludable & \textbf{Public} & Common\\
		Excludable & Collective & Private
	\end{tabular}
	\caption{Four Different Types of Goods}
\end{table}
\begin{description}
	\item[Rivalry:] The extent to which \textbf{consumption} of a good by one person lessens its availability for others
	\item[Non-rivalrous:] The extent to which the consumption of a good by one person does \textbf{not} lessen its availability for others
	\item[Excludability:] The extent to which non-payers can be excluded from consuming a good
	\item[Non-excludability:] Non-payers are able to consume the good
\end{description}
A public good demand is obtained by \textbf{vertically adding} each consumer's willingness to pay.